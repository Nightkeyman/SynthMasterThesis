\chapter*{SPIS TREŚCI} 
\contentsline {chapter}{Wykaz ważniejszych oznaczeń i skrótów}{8}{chapter*.3}%
\contentsline {chapter}{\numberline {1}Wstęp i cel pracy (Autor: Jan Sadłek)}{9}{chapter.1}%
\contentsline {chapter}{\numberline {2}Wprowadzenie teoretyczne (Autor: Jan Sadłek)}{11}{chapter.2}%
\contentsline {section}{\numberline {2.1}Dźwięk}{11}{section.2.1}%
\contentsline {section}{\numberline {2.2}Rodzaje klawiatur muzycznych}{11}{section.2.2}%
\contentsline {subsection}{\numberline {2.2.1}Klawiatura monofoniczna}{12}{subsection.2.2.1}%
\contentsline {subsection}{\numberline {2.2.2}Klawiatura polifoniczna}{12}{subsection.2.2.2}%
\contentsline {section}{\numberline {2.3}Przegląd metod syntezy dźwięku}{12}{section.2.3}%
\contentsline {subsection}{\numberline {2.3.1}Metody widmowe}{12}{subsection.2.3.1}%
\contentsline {subsection}{\numberline {2.3.2}Algorytmy abstrakcyjne}{12}{subsection.2.3.2}%
\contentsline {subsection}{\numberline {2.3.3}Metody fizyczne}{13}{subsection.2.3.3}%
\contentsline {subsection}{\numberline {2.3.4}Metody przetwarzania sygnału}{13}{subsection.2.3.4}%
\contentsline {section}{\numberline {2.4}Klasyczne moduły syntezatorów dźwięku}{13}{section.2.4}%
\contentsline {subsection}{\numberline {2.4.1}ADSR}{14}{subsection.2.4.1}%
\contentsline {section}{\numberline {2.5}Analogowa i cyfrowa synteza dźwięku}{14}{section.2.5}%
\contentsline {section}{\numberline {2.6}MIDI}{15}{section.2.6}%
\contentsline {subsection}{\numberline {2.6.1}Warstwa sprzętowa MIDI}{15}{subsection.2.6.1}%
\contentsline {subsection}{\numberline {2.6.2}Protokół MIDI}{16}{subsection.2.6.2}%
\contentsline {section}{\numberline {2.7}Dyskretna transformacja Fouriera}{16}{section.2.7}%
\contentsline {subsection}{\numberline {2.7.1}Definicja przekształcenia DFT}{17}{subsection.2.7.1}%
\contentsline {subsection}{\numberline {2.7.2}Właściwości przekształcenia DFT}{17}{subsection.2.7.2}%
\contentsline {subsection}{\numberline {2.7.3}Algorytm FFT}{18}{subsection.2.7.3}%
\contentsline {chapter}{\numberline {3}Projekt instrumentu muzycznego}{19}{chapter.3}%
\contentsline {section}{\numberline {3.1}Warstwa sprzętowa instrumentu (Autor: Jan Sadłek)}{19}{section.3.1}%
\contentsline {subsection}{\numberline {3.1.1}Płyta PADK}{19}{subsection.3.1.1}%
\contentsline {subsection}{\numberline {3.1.2}TMS320C6727}{20}{subsection.3.1.2}%
\contentsline {subsubsection}{\numberline {3.1.2.1}McASP}{20}{subsubsection.3.1.2.1}%
\contentsline {subsubsection}{\numberline {3.1.2.2}dMAX}{21}{subsubsection.3.1.2.2}%
\contentsline {subsection}{\numberline {3.1.3}Peryferia komunikacyjne}{21}{subsection.3.1.3}%
\contentsline {subsection}{\numberline {3.1.4}Przetworniki DAC}{21}{subsection.3.1.4}%
\contentsline {subsection}{\numberline {3.1.5}Pełen schemat układu instrumentu klawiszowego}{22}{subsection.3.1.5}%
\contentsline {section}{\numberline {3.2}Komunikacja z płytą PADK (Autor: Jakub Pełka)}{22}{section.3.2}%
\contentsline {subsection}{\numberline {3.2.1}Komunikacja z klawiaturą muzyczną (MIDI)}{22}{subsection.3.2.1}%
\contentsline {subsection}{\numberline {3.2.2}Komunikacja z interfejsem użytkownika (UART)}{24}{subsection.3.2.2}%
\contentsline {section}{\numberline {3.3}Generowanie sygnału analogowego (DAC) (Autor: Jan Sadłek)}{24}{section.3.3}%
\contentsline {subsection}{\numberline {3.3.1}Inicjalizacja programowa modułu DAC}{24}{subsection.3.3.1}%
\contentsline {subsection}{\numberline {3.3.2}Komunikacja między procesorem a przetwornikiem DAC}{25}{subsection.3.3.2}%
\contentsline {section}{\numberline {3.4}Klawiatura polifoniczna (Autor: Jakuba Pełka)}{25}{section.3.4}%
\contentsline {section}{\numberline {3.5}Wykorzystanie algorytmu FFT (Autor: Jan Sadłek)}{25}{section.3.5}%
\contentsline {subsection}{\numberline {3.5.1}Funkcje zawarte w pamięci ROM procesora}{26}{subsection.3.5.1}%
\contentsline {subsection}{\numberline {3.5.2}Pełna realizacja algorytmu przekształcenia}{26}{subsection.3.5.2}%
\contentsline {subsection}{\numberline {3.5.3}Rozdzielczość częstotliwości}{27}{subsection.3.5.3}%
\contentsline {section}{\numberline {3.6}ADSR (Autor: Jakub Pełka)}{27}{section.3.6}%
\contentsline {section}{\numberline {3.7}Interfejs użytkownika (Autor: Jakub Pełka)}{28}{section.3.7}%
\contentsline {chapter}{\numberline {4}Subtraktywna metoda syntezy dźwięku (Autor: Jakub Pełka)}{30}{chapter.4}%
\contentsline {section}{\numberline {4.1}Zasada działania syntezy subtraktywnej}{30}{section.4.1}%
\contentsline {section}{\numberline {4.2}Implementacja syntezy subtraktywnej}{33}{section.4.2}%
\contentsline {subsection}{\numberline {4.2.1}Generowanie przebiegu}{33}{subsection.4.2.1}%
\contentsline {subsection}{\numberline {4.2.2}Efekt Gibbsa}{34}{subsection.4.2.2}%
\contentsline {subsection}{\numberline {4.2.3}DFT i filtracja sygnału}{36}{subsection.4.2.3}%
\contentsline {subsection}{\numberline {4.2.4}Zakładkowanie bloków danych}{36}{subsection.4.2.4}%
\contentsline {subsection}{\numberline {4.2.5}Polifonia w syntezie subtraktywnej}{38}{subsection.4.2.5}%
\contentsline {section}{\numberline {4.3}Interfejs użytkownika}{39}{section.4.3}%
\contentsline {section}{\numberline {4.4}Wyniki}{40}{section.4.4}%
\contentsline {chapter}{\numberline {5}Addytywna synteza dźwięku (Autor: Jan Sadłek)}{41}{chapter.5}%
\contentsline {section}{\numberline {5.1}Zasada działania syntezy addytywnej}{41}{section.5.1}%
\contentsline {subsection}{\numberline {5.1.1}Postać harmoniczna}{41}{subsection.5.1.1}%
\contentsline {subsection}{\numberline {5.1.2}Postać nieharmoniczna}{42}{subsection.5.1.2}%
\contentsline {subsection}{\numberline {5.1.3}Składowe zmienne w czasie}{43}{subsection.5.1.3}%
\contentsline {subsection}{\numberline {5.1.4}Szum w syntezie addytywnej}{43}{subsection.5.1.4}%
\contentsline {section}{\numberline {5.2}Metody implementacji syntezy addytywnej}{43}{section.5.2}%
\contentsline {subsection}{\numberline {5.2.1}Ograniczenia syntezy addytywnej}{44}{subsection.5.2.1}%
\contentsline {subsection}{\numberline {5.2.2}Bank oscylatorów}{44}{subsection.5.2.2}%
\contentsline {subsection}{\numberline {5.2.3}Synteza wavetable}{45}{subsection.5.2.3}%
\contentsline {subsection}{\numberline {5.2.4}Synteza IFFT}{45}{subsection.5.2.4}%
\contentsline {section}{\numberline {5.3}Interfejs użytkownika}{46}{section.5.3}%
\contentsline {section}{\numberline {5.4}Realizacja organów na procesorze DSP}{47}{section.5.4}%
\contentsline {subsection}{\numberline {5.4.1}Opis implementacji}{47}{subsection.5.4.1}%
\contentsline {subsection}{\numberline {5.4.2}Wyniki}{47}{subsection.5.4.2}%
\contentsline {chapter}{\numberline {6}Synteza dźwięku - modulacja częstotliwości  (Autor: Jakub Pełka)}{49}{chapter.6}%
\contentsline {section}{\numberline {6.1}Zasada działania modulacji częstotliwości}{49}{section.6.1}%
\contentsline {subsection}{\numberline {6.1.1}Modulacja FM w dziedzinie czasu}{49}{subsection.6.1.1}%
\contentsline {subsection}{\numberline {6.1.2}Modulacja FM w dziedzinie częstotliwości}{50}{subsection.6.1.2}%
\contentsline {subsection}{\numberline {6.1.3}Rozbudowana modulacja FM}{52}{subsection.6.1.3}%
\contentsline {subsection}{\numberline {6.1.4}Projektowanie brzmień}{53}{subsection.6.1.4}%
\contentsline {section}{\numberline {6.2}Implementacja syntezy FM}{54}{section.6.2}%
\contentsline {subsection}{\numberline {6.2.1}Generowanie przebiegu}{54}{subsection.6.2.1}%
\contentsline {subsection}{\numberline {6.2.2}Polifonia w syntezie FM}{55}{subsection.6.2.2}%
\contentsline {section}{\numberline {6.3}Interfejs użytkownika}{55}{section.6.3}%
\contentsline {chapter}{\numberline {7}Modelowanie fizyczne}{57}{chapter.7}%
\contentsline {section}{\numberline {7.1}Synteza falowodowa (Autor: Jakub Pełka)}{57}{section.7.1}%
\contentsline {section}{\numberline {7.2}Synteza dźwięku skrzypiec (Autor: Jakub Pełka)}{58}{section.7.2}%
\contentsline {subsection}{\numberline {7.2.1}Zasada działania syntezy dźwięku skrzypiec}{58}{subsection.7.2.1}%
\contentsline {subsubsection}{\numberline {7.2.1.1}Rozchodzenie się fal w strunie}{59}{subsubsection.7.2.1.1}%
\contentsline {subsubsection}{\numberline {7.2.1.2}Interakcja pomiędzy smyczkiem a struną}{60}{subsubsection.7.2.1.2}%
\contentsline {subsubsection}{\numberline {7.2.1.3}Przechodzenie fal ze strun do korpusu skrzypiec}{61}{subsubsection.7.2.1.3}%
\contentsline {subsection}{\numberline {7.2.2}Implementacja syntezy dźwięku skrzypiec}{62}{subsection.7.2.2}%
\contentsline {subsection}{\numberline {7.2.3}Wyniki eksperymentalne}{64}{subsection.7.2.3}%
\contentsline {section}{\numberline {7.3}Synteza dźwięku fletu (Autor: Jan Sadłek)}{65}{section.7.3}%
\contentsline {subsection}{\numberline {7.3.1}Synteza falowodowa instrumentów dętych}{65}{subsection.7.3.1}%
\contentsline {subsubsection}{\numberline {7.3.1.1}Model}{65}{subsubsection.7.3.1.1}%
\contentsline {subsubsection}{\numberline {7.3.1.2}Implementacja}{66}{subsubsection.7.3.1.2}%
\contentsline {subsection}{\numberline {7.3.2}Synteza na podstawie modelu ARMA}{66}{subsection.7.3.2}%
\contentsline {subsubsection}{\numberline {7.3.2.1}Identyfikacja modelu}{67}{subsubsection.7.3.2.1}%
\contentsline {subsubsection}{\numberline {7.3.2.2}Parametryzacja zidentyfikowanego modelu}{68}{subsubsection.7.3.2.2}%
\contentsline {subsubsection}{\numberline {7.3.2.3}Implementacja}{69}{subsubsection.7.3.2.3}%
\contentsline {subsection}{\numberline {7.3.3}Wyniki}{69}{subsection.7.3.3}%
\contentsline {chapter}{\numberline {8}Podsumowanie (Autor: Jan Sadłek)}{71}{chapter.8}%
\contentsline {chapter}{Spis rysunk\'ow}{73}{chapter*.56}%
\contentsline {chapter}{Spis tabel}{75}{chapter*.57}%
\contentsline {chapter}{Bibliografia}{76}{chapter*.58}%
\contentsfinish 
