\chapter{Addytywna synteza dźwięku}\label{chapter_additive}
Synteza addytywna pozwala na utworzenie barwy dźwięku poprzez zbudowanie pełnego widma częstotliwościowego za pomocą odpowiednich składowych. Słowo 'addytywna' odnosi się do sumowania wielu przebiegów w jeden złożony sygnał dźwiękowy. Historycznie jest ona drugą najstarszą metodą syntezy, zaraz po subtraktywnej, opisanej w rozdziale \ref{chapter_subtractive}.
Zazwyczaj synteza addytywna polega na dodawaniu wielu fal sinusoidalnych o różnych częstotliwościach oraz amplitudach. Metoda ta pozwala na dużą dowolność. Dodawanymi składowywmi sygnału nie muszą być jedynie fale sinusoidalne.
Synteza addytywna uznawana jest za odwrotność syntezy subtraktywnej. Znajduje ona zastosowanie w syntezie dźwięku oraz mowy.

W niniejszym rozdziale przedstawiono zasadę działania syntezy addytywnej. Zaprezentowano wzory matematyczne, za pomocą których można uzyskać zsyntezowane brzmienie. Przedstawiono również kilka metod implementacji tego rodzaju syntezy. Na końcu rozdziału zaprezentowano interfejs użytkownika oraz wyniki zaimplementowania syntezy addytywnej na procesorze DSP.

\section{Zasada działania syntezy addytywnej}
Synteza addytywna jest zbliżona do analizy częstotliwościowej Fouriera. Z tego powodu jest ona zaliczana do widmowych metod syntezy. Jej postać matematyczną można jednak przedstawić w dziedzinie czasu. Zależnie od doboru składowych dźwięku, przedstawienie syntezy addytywnej może różnie wyglądać.
% Okresowosc sygnalu y(t)
%https://ccrma.stanford.edu/~jos/sasp/Additive_Synthesis_Early_Sinusoidal.html

\subsection{Postać harmoniczna} \label{pos_harm}
Dźwięk powstały w wyniku użycia syntezy addytywnej w formie harmonicznej można zapisać w postaci:
%https://en.wikipedia.org/wiki/Additive_synthesis#:~:text=Additive%20synthesis%20is%20a%20sound,or%20inharmonic%20partials%20or%20overtones.
\begin{equation} \label{equ:addit_time_harm}
y(t) = \sum_{k=1}^{K} a_{k}sin(2\pi kf_{0}t + \phi_{k})  \\  
\end{equation}
\begin{tabular}{ l l l l}
	gdzie: & $y(t)$ &  - & wyjście addytywnej syntezy dźwięku, \\
	&	$k$ & - &  numer składowej harmonicznej sygnału, \\
	&	$K$ & - &  całkowita liczba składowych harmonicznych dźwięku,\\
	&	$f_{0}$ & - &  częstotliwość pierwszej składowej harmonicznej,\\
	&	$a_{k}$ & - &  amplituda składowej harmonicznej k, \\
	&	$\phi_{k}$ & - &  faza składowej harmonicznej k. \\
\end{tabular}

Każda składowa dźwięku uzyskanego z tej postaci metody addytywnej jest wielokrotnością częstototliwości podstawowej $f_{0}$. Wykorzystanie takiej postaci syntezy addytywnej, pozwala na utworzenie dźwięku na przykład organów.

%https://en.wikibooks.org/wiki/Sound_Synthesis_Theory/Additive_Synthesis <<---- SCHEMAT 

%https://en.wikipedia.org/wiki/Square_wave
\begin{equation} \label{equ:addit_sqr}
y_{Square}(t) = \sum_{k=1}^{K} \frac{1}{2k-1} sin(2\pi (2k-1)f_{0}t) \\
\end{equation}

%https://en.wikipedia.org/wiki/Triangle_wave
\begin{equation} \label{equ:addit_trng}
y_{Triangle}(t) = \sum_{k=1}^{K} \frac{(-1)^k}{(2k-1)^2} sin(2\pi (2k-1)f_{0}t)  \\
\end{equation}

%https://en.wikipedia.org/wiki/Sawtooth_wave
\begin{equation} \label{equ:addit_sawth}
y_{Sawtooth}(t) = \sum_{k=1}^{K} \frac{(-1)^k}{k} sin(2\pi kf_{0}t) \\
\end{equation}
% O tym że na tym polegały organy Hammonda

Postać harmoniczna syntezy addytywnej pozwala również na uzyskanie podstawowych przebiegów używanych w syntezie subtraktywnej. Każdy z nich można wygenerowac za pomocą sumowania odpowiednich składowych harmonicznych z odpowiednimi amplitudami. Przykłady takich przebiegów przedstawiono we wzorach \ref{equ:addit_sqr}, \ref{equ:addit_trng} oraz \ref{equ:addit_sawth}.

\subsection{Postać nieharmoniczna} \label{pos_nieharm}
Dobieranie składowych dźwięku w syntezie addytywnej nie musi jednak zależeć od konkretnej częstotliwości. Niektóre instrumenty wydają dźwięk składający się ze składowych harmonicznych i nieharmonicznych (czyli takich, które nie są całkowitą wielokrotnością pewnej częstotliwości $f_{0}$). Dla takiej postaci metody addytywnej, zsyntezowany dźwięk można opisać wzorem:
\begin{equation} \label{equ:addit_time_nieharm}
y(t) = \sum_{k=1}^{K} a_{k}sin(2\pi f_{k}t + \phi_{k})  \\  
\end{equation}
\begin{tabular}{ l l l l}
	gdzie: 	&	$f_{k}$ & - &  częstotliwość składowej sygnału k,\\
\end{tabular}

Dźwięk opisany powyższym wzorem jest generowany przez instrumenty takie jak dzwony lub perkusjonalia.

\subsection{Składowe zmienne w czasie}
%https://books.google.pl/books/about/The_Computer_Music_Tutorial.html?id=nZ-TetwzVcIC&printsec=frontcover&source=kp_read_button&redir_esc=y#v=onepage&q&f=false
Wzory matematyczne \ref{equ:addit_time_harm} i \ref{equ:addit_time_nieharm} pozwalają na uzyskanie jedynie stanu ustalonego zsyntezowanego brzmienia. Powtarzany jest jeden okres sygnału, co daje wrażenie słuchaczowi, iż barwa dźwięku jest bardzo prosta.

Składowe dźwięku mogą jednak zmieniać się w czasie. Zmiany te mogą dotyczyć zarówno ich amplitudy, jak i częstotliwości. Synteza addytywna ze zmi
Taką postać syntezy addytywnej zapisuje się:
\begin{equation} \label{equ:addit_time_zmienne}
y(t) = \sum_{k=1}^{K} a_{k}(t)sin(2\pi f_{k}(t)t + \phi_{k})  \\  
\end{equation}
\begin{tabular}{ l l l l}
	gdzie: & $a_{k}(t)$ &  - & zmienna w czasie amplituda składowej k, \\
	&	$f_{k}(t)$ & - &  zmienna w czasie częstotliwość składowej k, \\
\end{tabular}

\subsection{Szum w syntezie addytywnej}
%https://ccrma.stanford.edu/~jos/sasp/S_N_Synthesis.html
Tworzenie sygnału zsyntezowanego metodą addytywną za pomocą wzorów przedstawionych powyżej jest deterministyczne. Do takiego sygnału można dodać jednak część stochastyczną. Uzyskuje się to z wykorzystaniem zmiennego w czasie filtra FIR oraz białego szumu.

\begin{equation} \label{equ:addit_szum}
B(\omega) = F(\omega)*e^{j\phi(\omega_{k})} \\  
\end{equation}
\begin{tabular}{ l l l l}
	gdzie: & $\phi(\omega_{k})$ &  - & faza losowa o rozkładzie równomiernym, \\
	& $F(\omega)$ &  - & obwiednia widmowa filtra FIR, \\
	&	$B(\omega)$ & - & zmienna w czasie częstotliwość składowej k, \\
\end{tabular}

Synteza części stochastycznej polega na wymnożeniu losowej fazy do obwiedni widma filtra. Takie działanie zostało zaprezentowane we wzorze \ref{equ:addit_szum}.
Dodanie szumu do sygnału deterministycznego syntezy addytywnej pozwala na uzyskanie dźwięków instrumentów dętych.

\section{Metody implementacji syntezy addytywnej}
%https://en.wikipedia.org/wiki/Additive_synthesis#History
Synteza addytywna w instrumentach klawiszowych używana jest między innymi do generacji dźwięku organów, których barwa upraszczana jest do kilku SH. W przypadku próby uzyskania dźwięku o większej ilości składowych harmonicznych, złożoność obliczeniowa metody wzrasta.
%https://ccrma.stanford.edu/~jos/pasp/Additive_Synthesis.html
Przykładowo, dla uzyskania pojedynczego dźwięku pianina w jakości CD-Audio,
%https://en.wikipedia.org/wiki/Compact_Disc_Digital_Audio
wymagane byłoby obliczenie około 400 fal sinusoidalnych na każdą próbkę zsyntezowanego dźwięku. Oznacza to, iż dla polifonicznej klawiatury takiego pianina, byłyby to tysiące SH. Obecne ograniczenia sprzętowe nie pozwalają na szybki rozwój addytywnej metody syntezy dźwięku, gdyż jest ona bardzo złożona obliczeniowo.

Istnieją różne metody implementacji addytywnej syntezy dźwięku. Ten sam dźwięk może być uzyskany różnymi sposobami. Popularne metody realizacji syntezy addytywnej to:
\begin{itemize}
	\item bank oscylatorów,
	\item synteza wavetable,
	\item synteza IFFT.
	% https://ieeexplore.ieee.org/document/4412805 86zł
\end{itemize}
W kolejnych punktach opisano wymienione metody implementacji syntezy addytywnej dźwięku.

\subsection{Bank oscylatorów}
%analogowe hammondy i zwykłe, ale ze są wolne

\subsection{Synteza wavetable}
%opisac tą która została zaimplementowana
%https://en.wikipedia.org/wiki/Lookup_table

\subsection{Synteza IFFT}
%Rysunki z matlaba


\section{Interfejs użytkownika}


\section{Wyniki}