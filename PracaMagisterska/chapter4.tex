\chapter{Addytywna synteza dźwięku}\label{chapter_additive}
Jak wspomniano w podrozdziale \ref{dzwiek}, brzmienie dźwięku jest zależne od ilości składowych harmonicznych. Dźwięki słyszane na codzień
Synteza addytywna polega na sumowaniu wielu fal sinusoidalnych w jedną. Każda ze składowych może mieć inną częstotliwość oraz amplitudę.
%https://en.wikipedia.org/wiki/Additive_synthesis#History
Opisywana metoda dźwięku znajduje zastosowanie w muzyce oraz syntezie mowy. Używana jest do generacji dźwięku różnego rodzaju organów. 

\section{Zasada działania syntezy addytywnej}
Istnieją różne podejścia do addytywnej syntezy dźwięku. Ten sam dźwięk może być uzyskany różnymi sposobami. Obecnie znane metody realizacji syntezy addytywnej to:
\begin{itemize}
	\item metoda czasowa - bank oscylatorów,
	\item synteza falowodowa,
	\item synteza IFFT.
	% https://ieeexplore.ieee.org/document/4412805
\end{itemize}

\subsection{Synteza addytywna czasowa}

%https://en.wikipedia.org/wiki/Additive_synthesis#:~:text=Additive%20synthesis%20is%20a%20sound,or%20inharmonic%20partials%20or%20overtones.
\begin{equation} \label{equ:addit_time}
y(t) = \sum_{k=1}^{K} r_{k}cos(2\pi kf_{0}t + \phi_{k})  \\  
\end{equation}
\begin{tabular}{ l l l l}
	gdzie: & $y(t)$ &  - & wyjście addytywnej syntezy dźwięku, \\
	&	$k$ & - &  numer składowej harmonicznej sygnału, \\
	&	$r_{k}$ & - &  amplituda składowej harmonicznej k, \\
	&	$kf_{0}$ & - &  częstotliwość składowej harmonicznej k,\\
	&	$\phi_{k}i$ & - &  faza składowej harmonicznej k, \\
\end{tabular}

\subsection{Synteza addytywna widmowa}



\section{Implementacja syntezy addytywnej}

\subsection{Synteza addytywna czasowa}

\subsection{Synteza addytywna widmowa}



\section{Interfejs użytkownika}


\section{Wyniki}