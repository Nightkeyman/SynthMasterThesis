\chapter{Subtraktywna metoda syntezy dźwięku}\label{chapter_subtractive}
\section{Zasada działania syntezy subtraktywnej}
Metoda subtraktywna polega na wygenerowaniu przebiegu okresowego bogatego pod względem składowych harmonicznych. Taki sygnał poddaje się filtracji w celu wytłumienia części składowych, co prowadzi do uzyskania interesujących dźwięków. Najczęściej wykorzystywanymi przebiegami w tej metodzie są: prostokątny, trójkątny, piłokształtny. Do usuwania składowych harmonicznych można zastosować filtry dolnoprzepustowe, górnoprzepustowego, pasmo-zaporowe czy pasmo-przepustowe. Na rysunku \ref{rys:sub_diagram} pokazano schemat blokowy opisanego postępowania.

\begin{figure}[H]
	\centering
	\includegraphics[width=12cm]{grafiki/sub_diagram}
	\captionsetup{justification=centering}
	\caption{Schemat blokowy dla metody subtraktywnej.}
	\label{rys:sub_diagram}
\end{figure}

Przykładowy wynik takiego działania przedstawiony jest na rysunku \ref{rys:sub_wykres1}.
\begin{figure}[H]
	\centering
	\includegraphics[width=12cm]{grafiki/sub_wykres1}
	\captionsetup{justification=centering}
	\caption{Zasada działania metody subtraktywnej.}
	\label{rys:sub_wykres1}
\end{figure}
Jak widać na \ref{rys:sub_wykres1}, na początku generowane jest 1024 próbek przbiegu prostokątnego. Następnie obliczane jest widmo tego sygnału. Z widma usuwane są wyższe składowe harmoniczne. Działanie to odpowiada filtracji idealnym filtrem dolnoprzepustowym. Z takiego widma, za pomocą odwrotnej transformacji Fouriera, obliczany jest przebieg czasowy. W wyniku otrzymuje się sygnał prostokątny po filtracji dolnoprzepustowej.
\section{Implementacja syntezy subtraktywnej}
W przypadku syntezatorów analogowych, poszczególne bloki są realizowane za pomocą elementów elektronicznych takich jak oscylatory (VCO) oraz przestrajalne filtry. W syntezatorach cyfrowych implementacje mogą być różne, jednak muszą być na tyle wydajne obliczeniowo, aby dana platforma sprzętowa poradziła sobie z syntezą dźwięku w czasie rzeczywistym bez artefaktów dźwiękowych. W tym podrozdziale opisana zostanie obrana droga implementacji metody subtraktywnej na procesorze sygnałowym. 
Próbki sygnału są syntezowane w sposób blokowy. Blok danych składa się z $N=1024$ próbek.

\subsection{Generowanie przebiegu}
W tablicy o $N$ elementach przechowywane są kolejne próbki sygnału bogatego w wyższe składowe harmoniczne. Po dokonaniu filtracji i wystawieniu tych próbek na przetwornik cyfrowo-analogowy, w tablicy tej należy wygenerować kolejne 1024 próbek sygnału. Trzeba przy tym pamiętać o odpowiednim przesunięciu fazowym, tak aby zachować ciągłość pomiędzy kolejnymi blokami próbek wystawianych na przetwornik. Równanie \ref{equ:sub_1} opisuje sposób generowania przebiegu prostokątnego.
\begin{equation} \label{equ:sub_1}
waveform[i]=\left \{\begin{array}{ r l }
1, & \quad \text{$dla$ } sin(2\pi f\frac{k+i}{F_s}) > 0\\
-1, & \quad  \text{$dla$ } sin(2\pi f\frac{k+i}{F_s}) \leqslant 0
\end{array}
\right.
\end{equation}
\begin{tabular}{ l l l l}
	gdzie: & $waveform$ &  - & tablica zmiennych typu float, \\
	&	$f$ & - &  pożądana częstotliwość generowanego przebiegu, \\
	&	$F_s$ & - & częstotliwość próbkowania,\\
	&	$i$ & - &  licznik iteracji, $i$ = 1, 2, 3, ..., N,\\
	&	$k$ & - &  licznik bloków.\\
\end{tabular} \\ \\
Po wyznaczeniu każdego bloku próbek, zwiększany jest licznik bloków $k$:
\begin{equation} \label{equ:sub_2}
k = k_{old} + N.
\end{equation}
Na rysunku \ref{rys:sub_waveform_blocks} pokazano dwa kolejne bloki próbek wygenerowanego przebiegu prostokątnego. Dzięki odpowiedniemu przesunięciu fazowemu przebiegi te, po wystawieniu na przetwornik cyfrowo-analogowy, utworzą ciągły dźwięk - bez trzasków.
\begin{figure}[H]
	\centering
	\includegraphics[width=12cm]{grafiki/sub_waveform_blocks}
	\captionsetup{justification=centering}
	\caption{Dwa kolejne bloki próbek przebiegu prostokątnego.}
	\label{rys:sub_waveform_blocks}
\end{figure}

\subsection{DFT i filtracja sygnału}
Na podstawie wygenerowanych próbek przebiegu oblicza się DFT za pomocą algorytmu FFT. W wyniku otrzymuje się $N$ liczb zespolonych. Przechowywane są one w tablicy o rozmiarze $2N$ w taki sposób, że każdy element o indeksie parzystym to część rzeczywista próbki, a sąsiadujący z nią element o indeksie nieparzystym to część urojona tej próbki. Filtracja dokonywana jest poprzez wyzerowanie elementów o odpowiednich indeksach. Na przykład filtracja dolnoprzepustowa z częstotliwością graniczną o wartości 1400 Hz dokonywana jest według poniżego wzoru:
\begin{equation} \label{equ:sub_3}
waveform_{fft}[i]=\left \{\begin{array}{ r l }
0, & \quad  \text{$dla$ } N - f_g \leqslant \frac{i}{2} \leqslant N + f_g \\
waveform_{fft}[i], & \quad \text{w pozostałych przypadkach } 

\end{array}
\right.
\end{equation}
\begin{tabular}{ l l l l}
	gdzie: & $waveform_{fft}$ &  - & próbki transformaty Fouriera sygnału, \\
	&	$freq$ & - &  częstotliwość przeliczona na indeksy w tablicy, $freq = N - 2 \frac{1400N}{Fs} - 1$, \\
	&	$F_s$ & - & częstotliwość próbkowania,\\
	&	$i$ & - &  indeks, $i$ = 1, 2, 3, ..., 2N.\\
\end{tabular} \\ \\

Efekt tego typu filtracji można zobaczyć na drugim i trzecim wykresie na \ref{rys:sub_wykres1}. W analogiczny sposób dokonuje się pozostałych filtracji: górnoprzepustowej, pasmo-przepustowej i pasmo-zaporowej.

\subsection{Zakładkowanie bloków danych}
Samo przesuwanie w fazie generowanych sygnałów opisanych w \ref{equ:sub_2} nie rozwiązuje wszystkich problemów. Jak widać na ostatnim wykresie na \ref{rys:sub_wykres1}, uzyskany sygnał na końcu ma niespodziewany przebieg. Po połączeniu tego bloku próbek z następnym blokiem otrzyma się sygnał pokazany na rysunku \ref{rys:sub_zakladkowania_brak}.
\begin{figure}[H]
	\centering
	\includegraphics[width=12cm]{grafiki/sub_zakladkowania_brak}
	\captionsetup{justification=centering}
	\caption{Dwa kolejne bloki próbek przebiegu prostokątnego.}
	\label{rys:sub_zakladkowania_brak}
\end{figure}
Rozwiązaniem tego problemu jest zakładkowanie każdych dwóch sąsiednich bloków próbek. Zakładkowanie polega na nasunięciu pewnej liczby $m$ początkowych próbek następnego bloku, na $m$ końcowych próbek bieżącego. W miejscach, gdzie bloki są na siebie nałożone, wartość każdej próbki obliczana jest jako średnia ważona z dwóch nałożonych na siebie próbek. Przy czym suma wag w każdej chwili czasu jest równa 1, wagi próbek bloku bieżącego maleją wraz z czasem, natomiast wagi próbek bloku następnego narastają. Najprostszym rozwiązaniem (a zatem efektywnym obliczeniowo) są wagi liniowe. Oznacza to, że wagi dla $m$ ostatnich lub pierwszych próbek bloku odpowiednio: liniowo maleją od 1 do 0 lub liniowo rosną od 0 do 1.
Nakładane na siebie próbki muszą odpowiadać sygnałowi w tej samej fazie, zatem należy zmodyfikować przesunięcie fazowe opisane przez \ref{equ:sub_2} do postaci \ref{equ:sub_4}:
\begin{equation} \label{equ:sub_4}
k = k_{old} + N - m.
\end{equation}
Porównanie zakładkowania dla różnych wartości $m$ przedstawiono na \ref{rys:sub_overlaps}.
\begin{figure}[H]
	\centering
	\includegraphics[width=10cm]{grafiki/sub_overlaps}
	\captionsetup{justification=centering}
	\caption{Wyniki zakladkowania przy zmiennej dlugosci zakladek.}
	\label{rys:sub_overlaps}
\end{figure}