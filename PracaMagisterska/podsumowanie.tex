\chapter{Podsumowanie}
% Czy udalo sie osiagnac cel pracy. Czy udalo sie spelnic te myslniki co mamy we wstepie
W niniejszej pracy zrealizowano zadania potrzebne do osiągnięcia celu. Zaprojektowano i wykonano instrument klawiszowy wykorzystujący syntezę subtraktywną, addytywną, FM oraz modelowanie fizyczne. Wymienione metody generowania dźwięku, zaimplementowano na procesorze sygnałowym TMS320C6727. Opracowano sterowanie transjentem dźwięku za pomocą klasycznej obwiedni ADSR. Kształt obwiedni widma zmieniano przy użyciu metody subtraktywnej. Interfejs użytkownika został zrealizowany w formie aplikacji na komputer PC.

% Synteza subtraktywna - porownanie do normalnego instrumentu. Wypada spoko. Filtry faktycznie filtrują. Trzeba uwazac na efekt gibbsa.

% Synteza addytywna - organy Hammonda brzmia dobrze. Moznaby dorzucic jeszcze efekty dzwiekowe. Ogolnie synteza jest ograniczona przez to ze duza zlozonosc obliczeniowa. w przyszlosci moze sie uda prawdziwe nasladowac nią z szumem np

% Synteza FM -- Udalo sie zbadac algorytmy modulujace fale sinusoidalne, dzieki ktorym uzyskano ciekawe brzmienia. Udalo sie uzyskac dzwon. CD>>>> Pelka??

% Synteza matematyczna -- Porownano dwa instrumenty pod wzgledem syntezy falowodowej. Udalo sie uzyskac dzwieki w symulacji. Na procesorze jest ciezko, duza zlozonosc obliczeniowa dla skomplikowanych modeli. A skomplikowane modele potrzebne zeby brzmienie bylo mozliwe do osiagniecia instrumentow roznych. Duza precyzja -- nie dzialalo na floatach.

% TRUDNOSCI:
%	- skompilkowana architektura płyty PADK
%	- slabe wsparcie ze strony producenta - jeden z nich nie istnieje
%	- przenoszenie kodu na DSP, wymaga optymalizacji
Osiągnięcie celu pracy wymagało od autorów umiejętności połączenia teorii z praktyką. Techniczny aspekt tej pracy, którym był projekt i realizacja instrumentu klawiszowego, wymagał zrozumienia architektury procesora oraz mechanizmów wspierających. W trakcie pracy okazało się również, iż firma produkująca płytę uruchomieniową PADK już nie istnieje.

% Czy ogolnie jakosc dzwiekow uzyskanych jest dobra
% Czy udalo sie zrobic lepszy syntezator niz rynkowy - no nie, bo to tylko prototyp.

