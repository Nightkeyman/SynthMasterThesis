\chapter{Podsumowanie}
% Czy udalo sie osiagnac cel pracy. Czy udalo sie spelnic te myslniki co mamy we wstepie
% Napisać, że celem pracy było:
% zbudowanie i bla bla bla - wypunktować
% Wszystkie cele opisane we wstępie niniejszej pracy zostały zrealizowane

Wszystkie cele projektu opisane we wstępie niniejszej pracy zostały zrealizowane. Zaprojektowano i wykonano prototyp instrumentu klawiszowego, wykorzystującego syntezę subtraktywną, addytywną, FM oraz modelowanie fizyczne. Wymienione metody generowania dźwięku zaimplementowano na procesorze sygnałowym TMS320C6727. Opracowano sterowanie transjentem dźwięku za pomocą klasycznej obwiedni ADSR. Kształt obwiedni widma zmieniano bezpośrednio przy użyciu metody subtraktywnej. Interfejs użytkownika został zrealizowany w formie aplikacji na komputerze PC. Zaimplementowano mechanizm komunikacji procesora z klawiaturą sterującą MIDI oraz z komputerem. 

Każda z metod syntezy dźwięku przedstawionych w tej pracy opiera się na innych regułach matematycznych. Synteza subtraktywna jest najstarszą z nich, stosowaną już na syntezatorach analogowych. Pozwala na zmianę brzmienia wykorzystując różne metody filtracji sygnału zawierającego wiele składowych harmonicznych. Synteza addytywna jest odwrotnością metody subtraktywnej. Daje możliwość utworzenia widma częstotliwościowego poprzez sumowanie wielu składowych. Niestety w chwili obecnej jest ograniczona wydajnością obliczeniową współczesnych procesorów. Istnieją jednak algorytmy pozwalające na usprawnienie efektywności tej metody, takie jak synteza IFFT. W pracy przedstawiono również metody generowania dźwięku z wykorzystaniem modulacji częstotliwości. Synteza FM pozwala osiągać bardzo ciekawe efekty dźwiękowe, które są często stosowane w muzyce elektronicznej.

Tworzenie dźwięku poprzez modelowanie fizyczne instrumentów jest bez wątpienia najbardziej skomplikowaną z przedstawionych w pracy metod. Opisuje za pomocą równań matematycznych lub schematów mechanikę fizyczną instrumentu oraz jego reakcję na pobudzenia instrumentalisty.
Synteza falowodowa charakteryzuje się stosunkowo niską złożonością obliczeniową. Dzięki temu jest często stosowanana do modelowania fizycznego instrumentów. Za jej pomocą uzyskuje się dobre efekty syntezy dźwięków muzycznych.
Autorzy przedstawili również syntezę fizyczną z wykorzystaniem modelu ARMA pobudzanego szumem białym. Metoda ta okazała się zbyt złożona obliczeniowo, a uzyskiwane dźwięki nie były najlepszej jakości.
Niestety z uwagi na ograniczone możliwości obliczeniowe procesorów DSP, nie można zaimplementować bardzo skomplikowanych mechanizmów fizycznych. Metoda modelowania instrumentów jest wciąż głównie używana w laboratoriach. Jedynie synteza falowodowa jest stosowana w rozwiązaniach komercyjnych.

Osiągnięcie celu pracy wymagało od autorów umiejętności połączenia teorii z praktyką. Techniczny aspekt tej pracy, którym był projekt i realizacja instrumentu klawiszowego, wymagał zrozumienia architektury procesora oraz mechanizmów przyspieszających przetwarzanie sygnałów. 
W~trakcie pracy nad dyplomem, okazało się, iż firma produkująca płytę uruchomieniową PADK już nie istnieje. Oznaczało to brak ewentualnego wsparcia ze strony producenta. Autorom udało się jednak zainicjalizować sprzęt oraz wykorzystać duże możliwości obliczeniowe procesora.

Dużym wyzwaniem było również przenoszenie kodu symulacji ze środowiska Matlab na DSP. Symulacja nie wymagała działania w czasie rzeczywistym, w przeciwieństwie do programu wykonywanego na procesorze. Dodatkowo pojawiały się takie problemy jak niedokładności obliczeń spowodowane użyciem zmiennych pojedynczej precyzji. Użycie zmiennych podwójnej precyzji tworzyło jednak problem większej złożoności obliczeniowej. Zagadnienie optymalizacji kodu miało bardzo duże znaczenie dla realizowanego projektu. Nawet małe niedokładności w obliczeniach kolejnych próbek oznaczały pojawienie się niepożądanych artefaktów dźwiękowych.

Obecnie instrumenty klawiszowe oferują sterowanie transjentami oraz obwiednią widma. Przeważnie posiadają one również klawiatury polifoniczne. Wciąż mogą jednak zostać rozwinięte na wiele różnych sposobów. Interfejsy użytkownika są często nieczytelne, a zsyntezowane brzmienia są dalekie od naturalnych barw instrumentów. Firmy zajmujące się rozwojem syntezatorów, powinny zwrócić szczególną uwagę na te dwa aspekty.

Kontynuacja badań na temat metod syntezy dźwięku powinna być skierowana głównie na modelowanie fizyczne instrumentów. Jako najmłodsza z metod pozostawia wiele nieodkrytych jeszcze możliwości kształtowania dźwieku i uzyskowania nowych brzmień. Utworzone modele mogą posiadać wiele parametrów, gdzie zmiana każdego z nich może mieć wpływ na barwę dźwięku o zupełnie innym charakterze. W celu osiągnięcia zadowalających rezultatów realizacji tak skomplikowanych metod syntezy, należy zwiększyć wydajność stosowanych procesorów DSP.
