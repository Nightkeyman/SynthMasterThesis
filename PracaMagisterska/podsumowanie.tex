\chapter{Podsumowanie}
% Czy udalo sie osiagnac cel pracy. Czy udalo sie spelnic te myslniki co mamy we wstepie

% Synteza subtraktywna - porownanie do normalnego instrumentu. Wypada spoko. Filtry faktycznie filtrują. Trzeba uwazac na efekt gibbsa.

% Synteza addytywna - organy Hammonda brzmia dobrze. Moznaby dorzucic jeszcze efekty dzwiekowe. Ogolnie synteza jest ograniczona przez to ze duza zlozonosc obliczeniowa. w przyszlosci moze sie uda prawdziwe nasladowac nią z szumem np

% Synteza FM -- Udalo sie zbadac algorytmy modulujace fale sinusoidalne, dzieki ktorym uzyskano ciekawe brzmienia. Udalo sie uzyskac dzwon. CD>>>> Pelka??

% Synteza matematyczna -- Porownano dwa instrumenty pod wzgledem syntezy falowodowej. Udalo sie uzyskac dzwieki w symulacji. Na procesorze jest ciezko, duza zlozonosc obliczeniowa dla skomplikowanych modeli. A skomplikowane modele potrzebne zeby brzmienie bylo mozliwe do osiagniecia instrumentow roznych. Duza precyzja -- nie dzialalo na floatach.

% TRUDNOSCI:
%	- skompilkowana architektura płyty PADK
%	- slabe wsparcie ze strony producenta - jeden z nich nie istnieje
%	- przenoszenie kodu na DSP, wymaga optymalizacji

% Czy ogolnie jakosc dzwiekow uzyskanych jest dobra
% Czy udalo sie zrobic lepszy syntezator niz rynkowy - no nie, bo to tylko prototyp.

