\chapter{Podsumowanie}
% Czy udalo sie osiagnac cel pracy. Czy udalo sie spelnic te myslniki co mamy we wstepie
Wszystkie zadania potrzebne do osiągnięcia celu niniejszej pracy zostały zrealizowane. Zaprojektowano i wykonano prototyp instrumentu klawiszowego wykorzystującego syntezę subtraktywną, addytywną, FM oraz modelowanie fizyczne. Wymienione metody generowania dźwięku zaimplementowano na procesorze sygnałowym TMS320C6727. Opracowano sterowanie transjentem dźwięku za pomocą klasycznej obwiedni ADSR. Kształt obwiedni widma zmieniano bezpośrednio przy użyciu metody subtraktywnej. Interfejs użytkownika został zrealizowany w formie aplikacji na komputerze PC. Zaimplementowano mechanizm komunikacji procesora z klawiaturą sterującą MIDI oraz z komputerem. 

Każda z metod syntezy dźwięku przedstawionych w tej pracy opiera się na innych regułach matematycznych. Synteza subtraktywna jest najstarszą z nich, stosowaną już na syntezatorach analogowych. Pozwala na zmianę brzmienia za pomocą filtrowania przebiegu czasowego posiadającego wiele składowych harmonicznych. Synteza addytywna jest odwrotnością metody subtraktywnej. Daję możliwość utworzenia widma częstotliwościowego poprzez sumowanie wielu składowych. Niestety jest ograniczana obecnymi możliwościami sprzętowymi procesorów. Istnieją jednak algorytmy pozwalające na usprawnienie efektywności tej metody, takie jak synteza IFFT. W pracy przedstawiono również generowanie dźwięku za pomocą modulacji jego częstotliwości. Synteza FM pozwala osiągać bardzo ciekawe efekty dźwiękowe, które są często stosowane w muzyce elektronicznej.
% Synteza subtraktywna - porownanie do normalnego instrumentu. Wypada spoko. Filtry faktycznie filtrują. Trzeba uwazac na efekt gibbsa.

% Synteza addytywna - organy Hammonda brzmia dobrze. Moznaby dorzucic jeszcze efekty dzwiekowe. Ogolnie synteza jest ograniczona przez to ze duza zlozonosc obliczeniowa. w przyszlosci moze sie uda prawdziwe nasladowac nią z szumem np

% Synteza FM -- Udalo sie zbadac algorytmy modulujace fale sinusoidalne, dzieki ktorym uzyskano ciekawe brzmienia. Udalo sie uzyskac dzwon. CD>>>> Pelka??

% Synteza matematyczna -- Porownano dwa instrumenty pod wzgledem syntezy falowodowej. Udalo sie uzyskac dzwieki w symulacji. Na procesorze jest ciezko, duza zlozonosc obliczeniowa dla skomplikowanych modeli. A skomplikowane modele potrzebne zeby brzmienie bylo mozliwe do osiagniecia instrumentow roznych. Duza precyzja -- nie dzialalo na floatach.
Synteza dźwięku poprzez modelowanie fizyczne instrumentów jest bez wątpienia najbardziej skomplikowaną z przedstawionych metod. Opisuję ona mechanikę fizyczną instrumentu oraz reakcję na pobudzenia instrumentalisty w postaci równań matematycznych lub schematów. Synteza falowodowa była przełomem dla metody modelowania fizycznego instrumentów pod względem uzyskiwanych efektów. Osiągnięto za jej pomocą bardzo zadowalające wyniki. Autorzy przedstawili również syntezę fizyczną poprzez pobudzenie modelu ARMA szumem, która okazała się być zbyt złożona obliczeniowo.
Niestety z uwagi na ograniczone możliwości obliczeniowe procesorów DSP, nie można zaimplementować bardzo skomplikowanych mechanizmów fizycznych. Metoda ta jest wciąż głównie używana w laboratoriach. Jedynie synteza falowodowa została wprowadzona na rynkowe instrumenty klawiszowe.

% TRUDNOSCI:
%	- skompilkowana architektura płyty PADK
%	- slabe wsparcie ze strony producenta - jeden z nich nie istnieje
%	- przenoszenie kodu na DSP, wymaga optymalizacji
Osiągnięcie celu pracy wymagało od autorów umiejętności połączenia teorii z praktyką. Postawiono przed nimi wiele wyzwań. Techniczny aspekt tej pracy, którym był projekt i realizacja instrumentu klawiszowego, wymagał zrozumienia architektury procesora oraz mechanizmów przyspieszających przetwarzanie sygnałów. 
W trakcie prac nad tym dyplomem, okazało się, iż firma produkująca płytę uruchomieniową PADK już nie istnieje. Oznaczało to brak ewentualnego wsparcia ze strony producenta. Autorom udało się jednak zainicjalizować sprzęt oraz wykorzystać duże możliwości obliczeniowe procesora.

Dużym wyzwaniem było również przenoszenie kodu z symulacji w środowisku Matlab na DSP. Symulacja nie wymagała działania w czasie rzeczywistym, w przeciwieństwie do programu na procesorze. Dodatkowo pojawiły się takie problemy jak niedokładności obliczeń spowodowane użyciem zmiennych pojedynczej precyzji. Użycie zmiennych podwójnej precyzji tworzyło jednak problem większej złożoności obliczeniowej. Zagadnienie optymalizacji kodu miało bardzo duże znaczenie dla realizowanego projektu. Nawet małe niedokładności w obliczeniach kolejnych próbek oznaczały pojawienie się niepożądanych artefaktów dźwiękowych.


% co nalezy zrobic do kontynuowania pracy
Obecnie instrumenty klawiszowe oferują sterowanie transjentami oraz obwiednią widma. Przeważnie posiadają one również klawiatury polifoniczne. Wciąż mogą jednak zostać rozwinięte na wiele różnych sposobów. Interfejsy użytkownika są często nieczytelne, a zsyntezowane brzmienia są dalekie od naturalnych barw instrumentów. Firmy zajmujące się rozwojem syntezatorów, powinny zwrócić szczególną uwagę na te dwa aspekty.

Kontynuacja badań na temat metod syntezy dźwięku powinna być skierowana głównie na modelowanie fizyczne instrumentów. Jako najmłodsza z metod pozostawia wiele nieodkrytych jeszcze obszarów. Utworzone modele mogą posiadać wiele parametrów, gdzie zmiana każdego z nich może tworzyć barwę dźwięku o zupełnie innym charakterze. Jednak aby osiągnięto zadowalające rezultaty implementacji tak skomplikowanych metod syntezy na DSP, najpierw należy pokonać ograniczenia sprzętowe. Optymalizacja algorytmów syntezy jest bardzo istotnym zagadnieniem w trakcie implementacji.
