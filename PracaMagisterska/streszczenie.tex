\chapter*{Streszczenie}
W ramach pracy wykonano prototyp syntezatora muzycznego. Platformę sprzętową stanowi płyta Professional Audio Development Kit (PADK) firmy Lyrtech wyposażona w procesor sygnałowy TMS320C6727 firmy Texas Instruments. Sterowanie pracą instrumentu zostało zrealizowane za pomocą klawiatury MIDI. Opisano i zaimplementowano następujące metody syntezy dźwięku: subtraktywną, addytywną, modulację FM oraz modelowania fizycznego. Utworzono graficzny interfejs użytkownika działający na komputerze klasy PC, który umożliwia zmianę parametrów generowanego dźwięku.
\newline
\\
\textbf{Słowa kluczowe}: instrument muzyczny, syntezator, synteza dźwięku, addytywna, subtraktywna, modelowanie fizyczne, cyfrowy falowód, modulacja częstotliwości, klawiatura midi, polifonia.
\\
\\
\textbf{Dziedzina nauki i techniki, zgodnie z wymogami OECD:} Nauki inżynieryjne i techniczne, elektrotechnika, elektronika, inżynieria informatyczna, elektrotechnika i elektronika.