\documentclass[nostrict]{szablonPG}


\usepackage{pdfpages}
\usepackage{multicol} % do sk³adu w dwóch kolumnach
\usepackage[nottoc]{tocbibind}
\usepackage[unicode=true]{hyperref}
\usepackage{graphicx}
\usepackage{float}
\usepackage{siunitx}
\usepackage{caption}
\raggedbottom
\usepackage[singlelinecheck=false % <-- important
]{caption}
\newcommand{\desctotoc}[1]{%
	\addtocontents{toc}{\medskip\noindent\detokenize{#1}\leavevmode\par\medskip}
}
\setcounter{secnumdepth}{4}
\begin{document}
\includepdf[pages=-]{StronaTytulowa_160648.pdf}
\includepdf[pages=-]{Oswiadczenie_160648.pdf}

	\setcounter{page}{3}
	\let\cleardoublepage\clearpage
	\chapter*{Streszczenie}
W ramach pracy wykonano prototyp syntezatora muzycznego. Platformę sprzętową stanowi płyta Professional Audio Development Kit (PADK) firmy Lyrtech wyposażona w procesor sygnałowy TMS320C6727 firmy Texas Instruments. Sterowanie pracą instrumentu zostało zrealizowane za pomocą klawiatury MIDI. Opisano i zaimplementowano następujące metody syntezy dźwięku: subtraktywną, addytywną, modulację FM oraz modelowania fizycznego. Utworzono graficzny interfejs użytkownika działający na komputerze klasy PC, który umożliwia zmianę parametrów generowanego dźwięku.
\newline
\\
\textbf{Słowa kluczowe}: instrument muzyczny, syntezator, synteza dźwięku, addytywna, subtraktywna, modelowanie fizyczne, cyfrowy falowód, modulacja częstotliwości, klawiatura midi, polifonia.
\\
\\
\textbf{Dziedzina nauki i techniki, zgodnie z wymogami OECD:} Nauki inżynieryjne i techniczne, elektrotechnika, elektronika, inżynieria informatyczna, sprzęt komputerowy i architektura komputerów.
	\chapter*{Abstract}
In this thesis a problem of state-space control of inverted pendulum was considered. For this purpose, equations describing behaviour of the plant where dervied using laws of physics. Also an experimental approach was made - a frequency response method - to obtain the model parameters. As the real control plant, a Bytronic pendulum module was used. An electronic circuit with microcontroller, for controlling the motor, measuring the cart's position and pendulum's angle, was designed and made. The control system was designed as a linear-quadratic regulator, implementing state feedback. A state observer was designed and implemented in order to obtain values of all plant's states. Effects of the design methods were successfully verified on the real inverted pendulum module.
\newline
\\
\textbf{Keywords}: inverted pendulum, state-space control, LQR regulator. 



	\tableofcontents    % spis treœci
	
%	\chapter*{SPIS TREŚCI} 
\contentsline {chapter}{Wykaz ważniejszych oznaczeń i skrótów}{8}{chapter*.3}%
\contentsline {chapter}{\numberline {1}Wstęp i cel pracy (Autor: Jan Sadłek)}{9}{chapter.1}%
\contentsline {chapter}{\numberline {2}Wprowadzenie teoretyczne (Autor: Jan Sadłek)}{11}{chapter.2}%
\contentsline {section}{\numberline {2.1}Dźwięk}{11}{section.2.1}%
\contentsline {section}{\numberline {2.2}Rodzaje klawiatur muzycznych}{11}{section.2.2}%
\contentsline {subsection}{\numberline {2.2.1}Klawiatura monofoniczna}{12}{subsection.2.2.1}%
\contentsline {subsection}{\numberline {2.2.2}Klawiatura polifoniczna}{12}{subsection.2.2.2}%
\contentsline {section}{\numberline {2.3}Przegląd metod syntezy dźwięku}{12}{section.2.3}%
\contentsline {subsection}{\numberline {2.3.1}Metody widmowe}{12}{subsection.2.3.1}%
\contentsline {subsection}{\numberline {2.3.2}Algorytmy abstrakcyjne}{12}{subsection.2.3.2}%
\contentsline {subsection}{\numberline {2.3.3}Metody fizyczne}{13}{subsection.2.3.3}%
\contentsline {subsection}{\numberline {2.3.4}Metody przetwarzania sygnału}{13}{subsection.2.3.4}%
\contentsline {section}{\numberline {2.4}Klasyczne moduły syntezatorów dźwięku}{13}{section.2.4}%
\contentsline {subsection}{\numberline {2.4.1}ADSR}{14}{subsection.2.4.1}%
\contentsline {section}{\numberline {2.5}Analogowa i cyfrowa synteza dźwięku}{14}{section.2.5}%
\contentsline {section}{\numberline {2.6}MIDI}{15}{section.2.6}%
\contentsline {subsection}{\numberline {2.6.1}Warstwa sprzętowa MIDI}{15}{subsection.2.6.1}%
\contentsline {subsection}{\numberline {2.6.2}Protokół MIDI}{16}{subsection.2.6.2}%
\contentsline {section}{\numberline {2.7}Dyskretna transformacja Fouriera}{16}{section.2.7}%
\contentsline {subsection}{\numberline {2.7.1}Definicja przekształcenia DFT}{17}{subsection.2.7.1}%
\contentsline {subsection}{\numberline {2.7.2}Właściwości przekształcenia DFT}{17}{subsection.2.7.2}%
\contentsline {subsection}{\numberline {2.7.3}Algorytm FFT}{18}{subsection.2.7.3}%
\contentsline {chapter}{\numberline {3}Projekt instrumentu muzycznego}{19}{chapter.3}%
\contentsline {section}{\numberline {3.1}Warstwa sprzętowa instrumentu (Autor: Jan Sadłek)}{19}{section.3.1}%
\contentsline {subsection}{\numberline {3.1.1}Płyta PADK}{19}{subsection.3.1.1}%
\contentsline {subsection}{\numberline {3.1.2}TMS320C6727}{20}{subsection.3.1.2}%
\contentsline {subsubsection}{\numberline {3.1.2.1}McASP}{20}{subsubsection.3.1.2.1}%
\contentsline {subsubsection}{\numberline {3.1.2.2}dMAX}{21}{subsubsection.3.1.2.2}%
\contentsline {subsection}{\numberline {3.1.3}Peryferia komunikacyjne}{21}{subsection.3.1.3}%
\contentsline {subsection}{\numberline {3.1.4}Przetworniki DAC}{21}{subsection.3.1.4}%
\contentsline {subsection}{\numberline {3.1.5}Pełen schemat układu instrumentu klawiszowego}{22}{subsection.3.1.5}%
\contentsline {section}{\numberline {3.2}Komunikacja z płytą PADK (Autor: Jakub Pełka)}{22}{section.3.2}%
\contentsline {subsection}{\numberline {3.2.1}Komunikacja z klawiaturą muzyczną (MIDI)}{22}{subsection.3.2.1}%
\contentsline {subsection}{\numberline {3.2.2}Komunikacja z interfejsem użytkownika (UART)}{24}{subsection.3.2.2}%
\contentsline {section}{\numberline {3.3}Generowanie sygnału analogowego (DAC) (Autor: Jan Sadłek)}{24}{section.3.3}%
\contentsline {subsection}{\numberline {3.3.1}Inicjalizacja programowa modułu DAC}{24}{subsection.3.3.1}%
\contentsline {subsection}{\numberline {3.3.2}Komunikacja między procesorem a przetwornikiem DAC}{25}{subsection.3.3.2}%
\contentsline {section}{\numberline {3.4}Klawiatura polifoniczna (Autor: Jakuba Pełka)}{25}{section.3.4}%
\contentsline {section}{\numberline {3.5}Wykorzystanie algorytmu FFT (Autor: Jan Sadłek)}{25}{section.3.5}%
\contentsline {subsection}{\numberline {3.5.1}Funkcje zawarte w pamięci ROM procesora}{26}{subsection.3.5.1}%
\contentsline {subsection}{\numberline {3.5.2}Pełna realizacja algorytmu przekształcenia}{26}{subsection.3.5.2}%
\contentsline {subsection}{\numberline {3.5.3}Rozdzielczość częstotliwości}{27}{subsection.3.5.3}%
\contentsline {section}{\numberline {3.6}ADSR (Autor: Jakub Pełka)}{27}{section.3.6}%
\contentsline {section}{\numberline {3.7}Interfejs użytkownika (Autor: Jakub Pełka)}{28}{section.3.7}%
\contentsline {chapter}{\numberline {4}Subtraktywna metoda syntezy dźwięku (Autor: Jakub Pełka)}{30}{chapter.4}%
\contentsline {section}{\numberline {4.1}Zasada działania syntezy subtraktywnej}{30}{section.4.1}%
\contentsline {section}{\numberline {4.2}Implementacja syntezy subtraktywnej}{33}{section.4.2}%
\contentsline {subsection}{\numberline {4.2.1}Generowanie przebiegu}{33}{subsection.4.2.1}%
\contentsline {subsection}{\numberline {4.2.2}Efekt Gibbsa}{34}{subsection.4.2.2}%
\contentsline {subsection}{\numberline {4.2.3}DFT i filtracja sygnału}{36}{subsection.4.2.3}%
\contentsline {subsection}{\numberline {4.2.4}Zakładkowanie bloków danych}{36}{subsection.4.2.4}%
\contentsline {subsection}{\numberline {4.2.5}Polifonia w syntezie subtraktywnej}{38}{subsection.4.2.5}%
\contentsline {section}{\numberline {4.3}Interfejs użytkownika}{39}{section.4.3}%
\contentsline {section}{\numberline {4.4}Wyniki}{40}{section.4.4}%
\contentsline {chapter}{\numberline {5}Addytywna synteza dźwięku (Autor: Jan Sadłek)}{41}{chapter.5}%
\contentsline {section}{\numberline {5.1}Zasada działania syntezy addytywnej}{41}{section.5.1}%
\contentsline {subsection}{\numberline {5.1.1}Postać harmoniczna}{41}{subsection.5.1.1}%
\contentsline {subsection}{\numberline {5.1.2}Postać nieharmoniczna}{42}{subsection.5.1.2}%
\contentsline {subsection}{\numberline {5.1.3}Składowe zmienne w czasie}{43}{subsection.5.1.3}%
\contentsline {subsection}{\numberline {5.1.4}Szum w syntezie addytywnej}{43}{subsection.5.1.4}%
\contentsline {section}{\numberline {5.2}Metody implementacji syntezy addytywnej}{43}{section.5.2}%
\contentsline {subsection}{\numberline {5.2.1}Ograniczenia syntezy addytywnej}{44}{subsection.5.2.1}%
\contentsline {subsection}{\numberline {5.2.2}Bank oscylatorów}{44}{subsection.5.2.2}%
\contentsline {subsection}{\numberline {5.2.3}Synteza wavetable}{45}{subsection.5.2.3}%
\contentsline {subsection}{\numberline {5.2.4}Synteza IFFT}{45}{subsection.5.2.4}%
\contentsline {section}{\numberline {5.3}Interfejs użytkownika}{46}{section.5.3}%
\contentsline {section}{\numberline {5.4}Realizacja organów na procesorze DSP}{47}{section.5.4}%
\contentsline {subsection}{\numberline {5.4.1}Opis implementacji}{47}{subsection.5.4.1}%
\contentsline {subsection}{\numberline {5.4.2}Wyniki}{47}{subsection.5.4.2}%
\contentsline {chapter}{\numberline {6}Synteza dźwięku - modulacja częstotliwości  (Autor: Jakub Pełka)}{49}{chapter.6}%
\contentsline {section}{\numberline {6.1}Zasada działania modulacji częstotliwości}{49}{section.6.1}%
\contentsline {subsection}{\numberline {6.1.1}Modulacja FM w dziedzinie czasu}{49}{subsection.6.1.1}%
\contentsline {subsection}{\numberline {6.1.2}Modulacja FM w dziedzinie częstotliwości}{50}{subsection.6.1.2}%
\contentsline {subsection}{\numberline {6.1.3}Rozbudowana modulacja FM}{52}{subsection.6.1.3}%
\contentsline {subsection}{\numberline {6.1.4}Projektowanie brzmień}{53}{subsection.6.1.4}%
\contentsline {section}{\numberline {6.2}Implementacja syntezy FM}{54}{section.6.2}%
\contentsline {subsection}{\numberline {6.2.1}Generowanie przebiegu}{54}{subsection.6.2.1}%
\contentsline {subsection}{\numberline {6.2.2}Polifonia w syntezie FM}{55}{subsection.6.2.2}%
\contentsline {section}{\numberline {6.3}Interfejs użytkownika}{55}{section.6.3}%
\contentsline {chapter}{\numberline {7}Modelowanie fizyczne}{57}{chapter.7}%
\contentsline {section}{\numberline {7.1}Synteza falowodowa (Autor: Jakub Pełka)}{57}{section.7.1}%
\contentsline {section}{\numberline {7.2}Synteza dźwięku skrzypiec (Autor: Jakub Pełka)}{58}{section.7.2}%
\contentsline {subsection}{\numberline {7.2.1}Zasada działania syntezy dźwięku skrzypiec}{58}{subsection.7.2.1}%
\contentsline {subsubsection}{\numberline {7.2.1.1}Rozchodzenie się fal w strunie}{59}{subsubsection.7.2.1.1}%
\contentsline {subsubsection}{\numberline {7.2.1.2}Interakcja pomiędzy smyczkiem a struną}{60}{subsubsection.7.2.1.2}%
\contentsline {subsubsection}{\numberline {7.2.1.3}Przechodzenie fal ze strun do korpusu skrzypiec}{61}{subsubsection.7.2.1.3}%
\contentsline {subsection}{\numberline {7.2.2}Implementacja syntezy dźwięku skrzypiec}{62}{subsection.7.2.2}%
\contentsline {subsection}{\numberline {7.2.3}Wyniki eksperymentalne}{64}{subsection.7.2.3}%
\contentsline {section}{\numberline {7.3}Synteza dźwięku fletu (Autor: Jan Sadłek)}{65}{section.7.3}%
\contentsline {subsection}{\numberline {7.3.1}Synteza falowodowa instrumentów dętych}{65}{subsection.7.3.1}%
\contentsline {subsubsection}{\numberline {7.3.1.1}Model}{65}{subsubsection.7.3.1.1}%
\contentsline {subsubsection}{\numberline {7.3.1.2}Implementacja}{66}{subsubsection.7.3.1.2}%
\contentsline {subsection}{\numberline {7.3.2}Synteza na podstawie modelu ARMA}{66}{subsection.7.3.2}%
\contentsline {subsubsection}{\numberline {7.3.2.1}Identyfikacja modelu}{67}{subsubsection.7.3.2.1}%
\contentsline {subsubsection}{\numberline {7.3.2.2}Parametryzacja zidentyfikowanego modelu}{68}{subsubsection.7.3.2.2}%
\contentsline {subsubsection}{\numberline {7.3.2.3}Implementacja}{69}{subsubsection.7.3.2.3}%
\contentsline {subsection}{\numberline {7.3.3}Wyniki}{69}{subsection.7.3.3}%
\contentsline {chapter}{\numberline {8}Podsumowanie (Autor: Jan Sadłek)}{71}{chapter.8}%
\contentsline {chapter}{Spis rysunk\'ow}{73}{chapter*.56}%
\contentsline {chapter}{Spis tabel}{75}{chapter*.57}%
\contentsline {chapter}{Bibliografia}{76}{chapter*.58}%
\contentsfinish 

	\addcontentsline{toc}{chapter}{Wykaz ważniejszych oznaczeń i skrótów}
\chapter*{Wykaz ważniejszych oznaczeń i skrótów}

\begin{tabular}{lcl}
	\textit{DSP} & -- & ang. digital signal processor -- procesor sygnałowy \\
	\textit{FFT} & -- & ang. Fast Fourier Transform -- szybka transformacja Fouriera \\
	\textit{USB} & -- & ang. Universal Serial Bus -- uniwersalna magistrala szeregowa \\
	\textit{MIDI} & -- & ang. Musical Instrument Digital Interface -- cyfrowy interfejs instrumentów muzycznych \\
  \textit{SH} & -- & składowa harmoniczna \\
	\textit{FM} & -- & ang. frequency modulation -- modulacja częstotliwości \\
	\textit{AM} & -- & ang. amplitude modulation -- modulacja amplitudy \\
	\textit{CS} & -- & cyfrowy syntezator \\
	\textit{AS} & -- & analogowy syntezator \\
	\textit{DFT} & -- & ang. Discrete Fourier Transform -- dyskretna transformata Fouriera \\
	\textit{VCO} & -- & ang. voltage-controlled oscillator -- oscylator kontrolowany napięciem \\
	\textit{VCA} & -- & ang. voltage-controlled amplifier -- wzmacniacz kontrolowany napięciem \\
	\textit{VCF} & -- & ang. voltage-controlled filter -- filtr kontrolowany napięciem \\
	\textit{LFO} & -- & ang. low-frequency oscillator -- generator wolnych przebiegów \\
	\textit{ADSR} & -- & ang. attack, delay, sustain, release -- generator obwiedni \\
	\textit{DAC} & -- & ang. digital-to-analog converter -- przetwornik cyfrowo-analogowy \\
	\textit{CPU} & -- & ang. central processing unit -- procesor \\
	\textit{UART} & -- & ang. universal asynchronous receiver-transmitter -- uniwersalny asynchroniczny nadajnik-odbiornik \\
	\textit{FPGA} & -- & ang. field-programmable gate array -- bezpośrednio programowalna macierz bramek \\
	\textit{JTAG} & -- & ang. Joint Test Action Group -- protokół testowania połączeń \\
	\textit{ISR} & -- & ang. interrupt service routine -- obsługa przerwania \\
  \textit{FIR} & -- & ang. finite impulse response -- skończona odpowiedź impulsowa \\
	\textit{GUI} & -- & ang. graphical user interface -- graficzny interfejs użytkownika \\
  \textit{STS} & -- & ang. short term spectrum -- krótkie okno widmowe \\
	\textit{IIR} & -- & ang. infinite impulse response -- nieskończona odpowiedź impulsowa \\
\end{tabular} 

	\chapter{Wstęp i cel pracy}
Muzyka jest ważnym aspektem życia każdego człowieka. Jest to dziedzina, która była rozwijana już od czasów prehistorycznych. %Napisac o instrumentach klasycznych. Instrumenty używały drgające struny, membrany , idiofony (marimba), Podac przyklady
%https://mlodytechnik.pl/eksperymenty-i-zadania-szkolne/wynalazczosc/29983-historia-wynalazkow-instrumenty-muzyczne
Na początku ludzie wykorzystywali proste instrumenty rytmiczne takie jak bębny czy grzechotki, które służyły do odprawiania rytuałów plemiennych. W starożytności wynaleziono instrumenty dęte oraz strunowe. W okresie średniowiecza i renesansu powstały instrumenty smyczkowe, chordofony (na przykład klawesyn) oraz aerofony (na przykład organy).
Wraz z rozwojem technologicznym, ludzie zaczęli używać skomplikowanych urządzeń do tworzenia muzyki. W pierwszej połowie XX wieku zaczęto używać instrumentów bazujących na układach scalonych. \cite{historia_instr}
% ogarnac po krótce budowe analogowych jak sie rozwijała
Na początku potrafiły one wydobywać z siebie proste dźwięki. Z czasem zaczęły posiadać coraz większe możliwości brzmieniowe. W drugiej połowie XX wieku cyfrowe instrumenty klawiszowe zaczęły być dostępne dla szerokiego kręgu odbiorców. Technologia cyfrowego przetwarzania sygnałów stworzyła nowe możliwości dla kompozytorów muzyki elektronicznej.

Syntezatory analogowe posiadały wiele niedoskonałości: nie pozwalały na precyzyjne odtworzenie pożądanych charakterystyk filtrów, miały ograniczone możliwości modelowania dźwięku oraz często z powodu starzenia się elementów elektronicznych zmieniały swoje parametry z upływem czasu. Dziedzina nauki zajmująca się syntezą dźwięku zaczęła dynamicznie rozwijać się wraz z rozwojem procesorów sygnałowych (DSP). Pojawienie się technologii cyfrowego przetwarzania sygnałów umożliwiło tworzenie dużo bardziej skomplikowanych brzmień. Procesory sygnałowe pozwalały na praktycznie nieograniczone zmiany charakterystyk generowanych dźwięków. Jednak zmiana sposobu syntezy z analogowej na cyfrową spowodowała pojawienie się również nowe ograniczenia. W systemach cyfrowych pojawił się problem doboru próbkowania sygnałów wejściowych wpływający na szerokość pasma ich przetwarzania.
Ważnym ograniczeniem była konieczność optymalizacji czasu wykonywania skomplikowanych algorytmów, wynikająca z konieczności ich zakończenia, do momentu pobrania kolejnej próbki.

Motywem realizacji tej pracy jest możliwość badania różnych metod syntezy dźwięku na procesorze sygnałowym. 
Autorzy zwrócili uwagę, iż w dotychczas spotykanych syntezatorach, próby imitacji brzmień naturalnych instrumentów, są dalekie od doskonałości.

Celem tej pracy było zrealizowanie polifonicznego instrumentu klawiszowego, który będzie wykorzystywał kilka typowych metod syntezy dźwięku. Realizacja projektu opierała się na zestawie uruchomieniowym Professional Audio Development firmy Lyrtech procesora sygnałowego TMS320C6727 oraz klawiaturze z interfejsem USB MIDI. Jednym z celów było również wykorzystanie dużych możliwości obliczeniowych zastosowanego procesora. Głównymi zadaniami do wykonania były:

\begin{enumerate}
    \item opracowanie i implementacja na procesorze sygnałowym algorytmów generowania wielotonów harmonicznych i dźwięków szumowych,
    
    \item opracowanie metod sterowania barwą generowanych dźwięków, dla różnych metod syntezy barw,
    
    \item realizacja interfejsu użytkownika.
\end{enumerate}
W niniejszej pracy omówiono realizację powyższych celów.

W rozdziale drugim przedstawiono informacje teoretyczne z zakresu teorii muzyki, podstaw syntezy dźwięku oraz cyfrowego przetwarzania sygnałów. W celu uzasadnienia wyboru cyfrowej syntezy dźwięku, przedstawiono porównanie wad i zalet metod analogowych i cyfrowych przetwarzania sygnałów. Informację tam zamieszczone są niezbędne do pełnego zrozumienia dalszej części pracy.

W rozdziale trzecim zaprezentowano implementację autorskiego instrumentu klawiszowego – sprzętową oraz programową. Opis działania programu oraz własności platformy sprzętowej pozwala na swobodną prezentację w dalszych rozdziałach pracy implementacji metod syntezy dźwięku. Zaprezentowano poszczególne moduły programu, w tym algorytm FFT na procesorze oraz metody sterowania protokołu MIDI. W tym samym rozdziale autorzy opisali realizację interfejsu instrumentu klawiszowego. Umożliwia on wybieranie odpowiedniej metody tworzenia dźwięku oraz dobranie parametrów ich syntezy. Interfejs zaimplementowano jako aplikację okienkową na komputerze PC, komunikującą się z programem działającym na procesorze sygnałowym.

W kolejnych czterech rozdziałach opisano metody syntezy dźwięków, które wykorzystywane są w instrumentach klawiszowych. Każdy z nich przedstawia teorię z zakresu danej metody, a następnie jej realizację praktyczną. Rozdział czwarty przedstawia najczęściej wykorzystywaną metodę syntezy – algorytm subtraktywny, który jest powszechnie wykorzystywany w syntezatorach analogowych. W rozdziale piątym zaprezentowano metodę addytywną oraz sposób generowania za jej pomocą dźwięku organów. W rozdziale szóstym omówiono syntezę dźwięku z użyciem modulacji częstotliwości (metoda FM). W rozdziale siódmym przedstawiono najbardziej skomplikowaną metodę syntezy omówioną w tej pracy – metodę modelowania matematycznego. Metoda ta wykorzystywana jest obecnie głównie w laboratoriach naukowych, gdyż daje największe możliwości uzyskania pożądanych brzmień syntezowanych dźwięków.

Rozdział ósmy jest podsumowaniem całej pracy. Omówiono w nim sposób realizacji przyjętych celów oraz przedstawiono wnioski końcowe.
	\chapter{Wprowadzenie teoretyczne}\label{chapter2}

W niniejszym rozdziale przedstawiono podstawowe pojęcia związane z dziedzinami muzyki, syntezy dźwięku oraz przetwarzania sygnałów cyfrowych. Wszystkie omówione zagadnienia odnoszą się do całości pracy. Nie są one powiązane szczególnie z konkretnymi metodami syntezy.



\section{Dźwięk}
Dźwięk można rozpatrywać na dwa sposoby: fizyczny oraz muzyczny. Z fizycznego punktu widzenia, dźwięk jest zaburzeniem falowym w ośrodku sprężystym gazowym, ciekłym lub stałym, który wywołuje wrażenie słuchowe u człowieka. Poddziedzina fizyki, zajmująca się ściśle tematem dźwięku, nazywana jest akustyką. % (źródło: https://encyklopedia.pwn.pl/haslo/dzwiek;3896050.html) 
W muzyce, dźwięk rozpatrywany jest jako zjawisko, które wydobywane jest z instrumentów muzycznych lub głosu ludzkiego. Główne właściwości dźwięku to:

\begin{enumerate}
	\item wysokość dźwięku, która jest zależna od wartości częstotliwości podstawowej dźwięku,
	
	\item czas trwania - zależny od czasu produkowanego dźwięku na danym instrumencie,
	
	\item głośność, zależna od amplitudy drgań powietrza przenoszącego dźwięk, mierzona w decybelach,
	
	\item barwa dźwięku, zależna od ilości, częstotliwości składowych harmonicznych dźwięku oraz zmian ich występowania w czasie. Jako składowa harmoniczna, rozumiana jest składowa sinusoidalna dźwięku, która jest całkowitą wielkrotnością częstotliwości podstawowej danego dźwięku.
	% (źródło: https://pl.wikipedia.org/wiki/D%C5%BAwi%C4%99k_(muzyka))
\end{enumerate}

Jako pewne ustandardyzowanie dźwięków w utworach muzycznych, wprowadzono skalę dźwięków. Tradycyjną skalę tworzy osiem dźwięków. Odległości między tymi dźwiękami nazywane są interwałami, które wyraża się w jednostkach półtonów. 

Odstęp między pierwszym a ostatnim dźwiękiem w skali nazywany jest oktawą. Mierzy on 12 półtonów. Odległość ta jest wyjątkowa, gdyż dźwięk położony o oktawę dalej od pierwszego, jest jego dwukrotnością pod względem częstotliwości podstawowej składowej harmonicznej.

Istnieje również zależność między kolejnymi półtonami. Jest ona wyrażona wzorem:



\section{Rodzaje klawiatur muzycznych}
Podstawowy podział klawiatur muzycznych rozpatruje się pod względem możliwości wydawania z siebie jednego lub kilku dźwięków przy naciśnięciu kilku klawiszy na raz. Typy te nazwano klawiaturami monofonicznymi oraz polifonicznymi.

Monofonia w muzyce to faktura muzyczna utworzona z pojedynczej linii melodycznej. W danej chwili czasu utworu, powinien występować tylko jeden dźwięk, aby był on nazwany monofonicznym. W szczególnym odniesieniu do klawiatury cyfrowej lub analogowej oznacza to, iż wydobywać się z niej może się maksymalnie jeden dźwięk w jednym momencie czasu, przy naciśnięciu kilku klawiszy. Klasycznym przykładem klawiatury monofonicznej jest syntezator analogowy Minimoog.

Polifonia w muzyce oznacza natomiast występowanie kilku linii melodycznych w tym samym czasie. Polifonia klawiatury zatem oznacza możliwość wydobycia wielu dźwięków, przy naciśnięciu kilku klawiszy w tym samym czasie. Pojęcie polifonicznej klawiatury muzycznej nie precyzuje jaką ilośc wydobywających się dźwięków klawiatura powinna mieć możliwość realizacji. Cyfrowe instrumenty muzyczne na rynku posiadają ograniczoną maksymalną liczbę wydobywanych na raz dźwięków, przez skończoną moc obliczeniową procesorów.

%
% Voice allocation alghorithm - opisac jaki bedziemy uzywac
%

Jednym z głównych zadań niniejszego projektu magisterskiego jest implementacja polifonicznej klawiatury muzycznej.



\section{Przegląd metod syntezy dźwięku}
% http://legacy.spa.aalto.fi/publications/reports/sound_synth_report.pdf
Podział metod syntezy dźwięku w literaturze jest bardzo zróżnicowany. Najczęściej spotykany podział na metody syntezy ze względu na różne podejścia to:
\begin{enumerate}
	\item metody widmowe,
	\item algorytmy abstrakcyjne,
	\item metody fizyczne,
	\item metody przetwarzania sygnału,
	% https://ieeexplore.ieee.org/document/4412805
\end{enumerate}

W poniższych podrozdziałach scharakteryzowano poszczególne rodzaje metod syntezy dźwięku.

\subsection{Metody widmowe}
Do grupy widmowych metod syntezy dźwięku zaliczana jest synteza subtraktywna oraz addytywna. Zgodnie z nazwą, skupiają się one na syntezie brzmień w dziedzinie częstotliwości. Metoda subtraktywna zajmuje się głównie wycinaniem pasm widmowych z brzmień obfitych w składowe harmoniczne. Metoda addytywna skupia się natomiast na dodawaniu kolejnych harmonicznych w dziedzinie częstotliwości. Obie metody zostały dokładnie omówione w kolejnych rozdziałach [jakie rozdziały? - zrobic odnosnik] niniejszej pracy.

\subsection{Algorytmy abstrakcyjne}
% https://ccrma.stanford.edu/~bilbao/booktop/node6.html
% http://legacy.spa.aalto.fi/publications/reports/sound_synth_report.pdf
Synteza poprzez algorytmy abstrakcyjne zazwyczaj odnosi się do zmian istniejących już dźwięków poprzez filtry nieliniowe lub funkcje matematyczne. Do metod algorytmów abstrakcyjnych należą między innymi synteza FM, synteza AM, synteza poprzez waveshaping lub algorytm Karplus-Strong. W literaturze można również spotkać się z określaniem tej grupy metod syntezy jako Distortion synthesis, która odnosi się do wprowadzanych zmian w istniejącym już dźwięku (zakłóceń). W niniejszej pracy poświęcono cały rozdział dla pierwszej z wymienionych metod.

\subsection{Metody fizyczne}
% https://ccrma.stanford.edu/~bilbao/booktop/node12.html
% https://edu.pjwstk.edu.pl/wyklady/mul/scb/main37.html
Metody fizyczne skupiają się na odwzorowywaniu instrumentów muzycznych poprzez tworzenie ich modeli fizycznych na różne sposoby. Do tej grupy należą synteza poprzez modelowanie matematyczne, synteza komórkowa (ang. Cellular Sound Synthesis) oraz synteza falowodowa (ang. Digital Waveguide Modeling).

Metody syntezy oparte na modelach fizycznych są uważane za najbardziej naukowe i poświęcone jest im wiele prac naukowych. W niniejszej pracy przedstawiono głównie synteza opartą o modelowanie matematyczne.

\subsection{Metody przetwarzania sygnału}
%http://marcdata.hamu.cz/vyzkum/dokumenty/Lit92.pdf - additive, subtractive, FM. sampling
%https://soundlab.cs.princeton.edu/publications/survey_icmc09.pdf - 4 strona, tabelka
%https://www.youtube.com/watch?v=I64y40EIPaM
Jako metody przetwarzania sygnału rozumiane są takie metody syntezy, które oddziaływują bezpośrednio na próbkach sygnału cyfrowego. Do tej grupy metod należą między innymi synteza tablicowa (wavetable), synteza granularna oraz synteza poprzez samplowanie.

Synteza poprzez samplowanie jest najbardziej powszechną metodą syntezy dźwięku w tej grupie. Polega ona na użyciu fragmentu wcześniej dokonanego nagrania (sampla) i odtwarzanie go dla różnych wysokości dźwięków. Jest to metoda najczęściej używana przez kompozytorów w dzisiejszych czasach.

Z uwagi na nieunaukowy charakter tego rodzaju metody syntezy dźwięku, żadna synteza należąca do tej grupy nie została opisana w tej pracy.



\section{Klasyczne moduły syntezatorów dźwięku}
%https://www.dummies.com/art-center/music/piano/common-keyboard-terms-and-abbreviations/
%https://en.wikipedia.org/wiki/Modular_synthesizer#:~:text=Modular%20synthesizers%20are%20synthesizers%20composed,user%20to%20create%20a%20patch.
\subsection{Obwiednia dźwięku (ADSR)}




\section{Analogowa i cyfrowa synteza dźwięku}




\section{MIDI}


\subsection{Warstwa sprzętowa MIDI}

\subsection{Protokół MIDI}
%https://www.midi.org/specifications-old/category/midi-1-0-detailed-specifications
%PDF: D:\Data\Studia\Magisterka\Literatura wstepna\MIDI




\section{Dyskretna transformata Fouriera}
Jednym z najważniejszych narzędzi w dziedzinie cyfrowego przetwarzania sygnałów jest dyskretna transformata Fouriera. 
\subsection{Algorytm FFT}

	% !TeX spellcheck = pl_PL
\chapter{Realizacja instrumentu muzycznego}
Implementacja tworzonego syntezatora muzycznego realizowana jest na wspomnianej we wstępie płycie PADK, która opiera swoje działanie na procesorze DSP. W ramach tego implementowane jest wiele mechanizmów, które są konieczne do kompletnego działania projektu: przetwarzanie sygnałów z klawiatury muzycznej, wydobycie dźwięku z płyty PADK oraz komunikacja z interfejsem użytkownika. W niniejszym rozdziale przedstawiono podejście do realizacji wymienionych zadań. Implementacja poszczególnych algorytmów syntezy dźwięku przedstawiona zostanie w kolejnych rozdziałach.

Cały kod programu na procesor DSP napisany został w języku C. Program pisano w środowisku Code Composer Studio v6, które jest dedykowane do procesorów firmy Texas Instruments. Wgrywanie kodu na płytę odbywało się poprzez użycie debuggera XDS510 firmy Spectrum Digital. Debugger połączony jest z płytą PADK przez taśmę, a następnie przejściówkę 8-pinową.

% zdjęcie PADK z debuggerem

\section{Parametry procesora TMS320C6727}
%https://www.ti.com/lit/ds/symlink/tms320c6727.pdf?ts=1595978326019&ref_url=https%253A%252F%252Fwww.google.com%252F --> strona 13 wypisane w myslnikach np busy

\section{Mechanizmy szybkiego przetwarzania danych}
Procesor sygnałowy TMS320C6727 jest przeznaczony między innymi do szybkiego przetwarzania danych i sygnałów. Poza wysoką szybkością taktowania procesora, firma Texas Instruments zawiera dodatkowe mechanizmy akceleracji przepływu danych i sygnałów. Zostały one opisane w poniższych podpunktach.

\subsection{McASP}
% https://www.ti.com/lit/an/sprack0/sprack0.pdf?ts=1595842361762&ref_url=https%253A%252F%252Fwww.google.com%252F
McASP to akronim od Multichannel Audio Serial Port. Jest to komunikacyjne urządzenie peryferyjne dedykowane do przetwarzania danych audio lub wideo. Został zaprojektowany w celu przypadków wymagających wielokanałowego przetwarzania dźwięku.

Jedną z najbardziej przydatnych rzeczy w odniesieniu do narzędzia McASP jest schemat wielozegarowy. Pozwala on na niezależność pomiędzy portami odbierającymi i nadającymi

\subsection{dMAX}
%https://www.ti.com/lit/ug/spru795d/spru795d.pdf?ts=1595843361787&ref_url=https%253A%252F%252Fwww.google.com%252F, strona 14, Overview
Kontroler dMAX (Dual Data Movement Accelerator) obsługuje transfery zaprogramowane przez użytkownika pomiędzy kontrolerem pamięci wewnętrznej i urządzeniami peryferialnymi na procesorach DSP firmy TI. Mechanizm ten jest dedykowany szczególnie dla procesorów z serii C672x.

Zasada działania dMAX opiera się na sygnałach zdarzeń (ang. event signals). Zdarzenie zdefiniowane jest jako zmiana wartości logicznej odpowiadającego sygnału zdarzeń w rejestrze flag zdarzeń. Zdarzenie może być używane jako: wzbudzenie rozpoczęcia transferu danych lub spowodowanie wystąpienia przerwania dla CPU. Wszystkie zdarzenia posortowane są w dwie grupy: grupa niskiego priorytetu oraz grupa wysokiego priorytetu. Mechanizm dMAX może równolegle przetwarzać dwa żądania zdarzeń z każdej z grupy.

Częścią mechanizmu dMAX jest również bufor cyrkulacyjny FIFO. Pozwala on na równoczesny, asynchroniczny odczyt i zapis danych do jednego bufora dwustronnego. Narzędzie dMAX wykrywa kiedy dane zostają zapisane do bufora i natychmiastowo wywołuje odwrócenie go. Po odwróceniu, zapisane chwilę wcześniej dane mogą zostać odczytane z drugiej strony bufora, natomiast równocześnie kolejne dane zapisywane są po pierwszej stronie.


\section{Komunikacja z klawiaturą muzyczną  (MIDI)}




\section{Komunikacja z interfejsem użytkownika (UART)}




\section{Interfejs użytkownika}




\section{Wydobycie dźwięku (DAC)}




\section{Klawiatura polifoniczna}




\section{Implementacja FFT}
% jakl linkujemy biblioteke, o bibliotece
% bitreverse opisac to co mamy


\section{Generowanie przebiegów czasowych}



\section{ADSR}
	\chapter{Subtraktywna metoda syntezy dźwięku}\label{chapter_subtractive}
\section{Zasada działania syntezy subtraktywnej}
Metoda subtraktywna polega na wygenerowaniu przebiegu okresowego bogatego pod względem składowych harmonicznych. Taki sygnał poddaje się filtracji w celu wytłumienia części składowych, co prowadzi do uzyskania interesujących dźwięków. Najczęściej wykorzystywanymi przebiegami w tej metodzie są: prostokątny, trójkątny, piłokształtny. Do usuwania składowych harmonicznych można zastosować filtry dolnoprzepustowe, górnoprzepustowego, pasmo-zaporowe czy pasmo-przepustowe. Na rysunku \ref{rys:sub_diagram} pokazano schemat blokowy opisanego postępowania.

\begin{figure}[H]
	\centering
	\includegraphics[width=12cm]{grafiki/sub_diagram}
	\captionsetup{justification=centering}
	\caption{Schemat blokowy dla metody subtraktywnej.}
	\label{rys:sub_diagram}
\end{figure}

Przykładowy wynik takiego działania przedstawiony jest na rysunku \ref{rys:sub_wykres1}.
\begin{figure}[H]
	\centering
	\includegraphics[width=12cm]{grafiki/sub_wykres1}
	\captionsetup{justification=centering}
	\caption{Zasada działania metody subtraktywnej.}
	\label{rys:sub_wykres1}
\end{figure}
Jak widać na \ref{rys:sub_wykres1}, na początku generowane jest 1024 próbek przbiegu prostokątnego. Następnie obliczane jest widmo tego sygnału. Z widma usuwane są wyższe składowe harmoniczne. Działanie to odpowiada filtracji idealnym filtrem dolnoprzepustowym. Z takiego widma, za pomocą odwrotnej transformacji Fouriera, obliczany jest przebieg czasowy. W wyniku otrzymuje się sygnał prostokątny po filtracji dolnoprzepustowej.
\section{Implementacja syntezy subtraktywnej}
W przypadku syntezatorów analogowych, poszczególne bloki są realizowane za pomocą elementów elektronicznych takich jak oscylatory (VCO) oraz przestrajalne filtry. W syntezatorach cyfrowych implementacje mogą być różne, jednak muszą być na tyle wydajne obliczeniowo, aby dana platforma sprzętowa poradziła sobie z syntezą dźwięku w czasie rzeczywistym bez artefaktów dźwiękowych. W tym podrozdziale opisana zostanie obrana droga implementacji metody subtraktywnej na procesorze sygnałowym. 
Próbki sygnału są syntezowane w sposób blokowy. Blok danych składa się z $N=1024$ próbek.

\subsection{Generowanie przebiegu}
W tablicy o $N$ elementach przechowywane są kolejne próbki sygnału bogatego w wyższe składowe harmoniczne. Po dokonaniu filtracji i wystawieniu tych próbek na przetwornik cyfrowo-analogowy, w tablicy tej należy wygenerować kolejne 1024 próbek sygnału. Trzeba przy tym pamiętać o odpowiednim przesunięciu fazowym, tak aby zachować ciągłość pomiędzy kolejnymi blokami próbek wystawianych na przetwornik. Równanie \ref{equ:sub_1} opisuje sposób generowania przebiegu prostokątnego.
\begin{equation} \label{equ:sub_1}
waveform[i]=\left \{\begin{array}{ r l }
1, & \quad \text{$dla$ } sin(2\pi f\frac{k+i}{F_s}) > 0\\
-1, & \quad  \text{$dla$ } sin(2\pi f\frac{k+i}{F_s}) \leqslant 0
\end{array}
\right.
\end{equation}
\begin{tabular}{ l l l l}
	gdzie: & $waveform$ &  - & tablica zmiennych typu float, \\
	&	$f$ & - &  pożądana częstotliwość generowanego przebiegu, \\
	&	$F_s$ & - & częstotliwość próbkowania,\\
	&	$i$ & - &  licznik iteracji, $i$ = 1, 2, 3, ..., N,\\
	&	$k$ & - &  licznik bloków.\\
\end{tabular} \\ \\
Po wyznaczeniu każdego bloku próbek, zwiększany jest licznik bloków $k$:
\begin{equation} \label{equ:sub_2}
k = k_{old} + N.
\end{equation}
Na rysunku \ref{rys:sub_waveform_blocks} pokazano dwa kolejne bloki próbek wygenerowanego przebiegu prostokątnego. Dzięki odpowiedniemu przesunięciu fazowemu przebiegi te, po wystawieniu na przetwornik cyfrowo-analogowy, utworzą ciągły dźwięk - bez trzasków.
\begin{figure}[H]
	\centering
	\includegraphics[width=12cm]{grafiki/sub_waveform_blocks}
	\captionsetup{justification=centering}
	\caption{Dwa kolejne bloki próbek przebiegu prostokątnego.}
	\label{rys:sub_waveform_blocks}
\end{figure}

\subsection{DFT i filtracja sygnału}
Na podstawie wygenerowanych próbek przebiegu oblicza się DFT za pomocą algorytmu FFT. W wyniku otrzymuje się $N$ liczb zespolonych. Przechowywane są one w tablicy o rozmiarze $2N$ w taki sposób, że każdy element o indeksie parzystym to część rzeczywista próbki, a sąsiadujący z nią element o indeksie nieparzystym to część urojona tej próbki. Filtracja dokonywana jest poprzez wyzerowanie elementów o odpowiednich indeksach. Na przykład filtracja dolnoprzepustowa z częstotliwością graniczną o wartości 1400 Hz dokonywana jest według poniżego wzoru:
\begin{equation} \label{equ:sub_3}
waveform_{fft}[i]=\left \{\begin{array}{ r l }
0, & \quad  \text{$dla$ } N - f_g \leqslant \frac{i}{2} \leqslant N + f_g \\
waveform_{fft}[i], & \quad \text{w pozostałych przypadkach } 

\end{array}
\right.
\end{equation}
\begin{tabular}{ l l l l}
	gdzie: & $waveform_{fft}$ &  - & próbki transformaty Fouriera sygnału, \\
	&	$freq$ & - &  częstotliwość przeliczona na indeksy w tablicy, $freq = N - 2 \frac{1400N}{Fs} - 1$, \\
	&	$F_s$ & - & częstotliwość próbkowania,\\
	&	$i$ & - &  indeks, $i$ = 1, 2, 3, ..., 2N.\\
\end{tabular} \\ \\

Efekt tego typu filtracji można zobaczyć na drugim i trzecim wykresie na \ref{rys:sub_wykres1}. W analogiczny sposób dokonuje się pozostałych filtracji: górnoprzepustowej, pasmo-przepustowej i pasmo-zaporowej.

\subsection{Zakładkowanie bloków danych}
Samo przesuwanie w fazie generowanych sygnałów opisanych w \ref{equ:sub_2} nie rozwiązuje wszystkich problemów. Jak widać na ostatnim wykresie na \ref{rys:sub_wykres1}, uzyskany sygnał na końcu ma niespodziewany przebieg. Po połączeniu tego bloku próbek z następnym blokiem otrzyma się sygnał pokazany na rysunku \ref{rys:sub_zakladkowania_brak}.
\begin{figure}[H]
	\centering
	\includegraphics[width=12cm]{grafiki/sub_zakladkowania_brak}
	\captionsetup{justification=centering}
	\caption{Dwa kolejne bloki próbek przebiegu prostokątnego.}
	\label{rys:sub_zakladkowania_brak}
\end{figure}
Rozwiązaniem tego problemu jest zakładkowanie każdych dwóch sąsiednich bloków próbek. Zakładkowanie polega na nasunięciu pewnej liczby $m$ początkowych próbek następnego bloku, na $m$ końcowych próbek bieżącego. W miejscach, gdzie bloki są na siebie nałożone, wartość każdej próbki obliczana jest jako średnia ważona z dwóch nałożonych na siebie próbek. Przy czym suma wag w każdej chwili czasu jest równa 1, wagi próbek bloku bieżącego maleją wraz z czasem, natomiast wagi próbek bloku następnego narastają. Najprostszym rozwiązaniem (a zatem efektywnym obliczeniowo) są wagi liniowe. Oznacza to, że wagi dla $m$ ostatnich lub pierwszych próbek bloku odpowiednio: liniowo maleją od 1 do 0 lub liniowo rosną od 0 do 1.
Nakładane na siebie próbki muszą odpowiadać sygnałowi w tej samej fazie, zatem należy zmodyfikować przesunięcie fazowe opisane przez \ref{equ:sub_2} do postaci \ref{equ:sub_4}:
\begin{equation} \label{equ:sub_4}
k = k_{old} + N - m.
\end{equation}
Porównanie zakładkowania dla różnych wartości $m$ przedstawiono na \ref{rys:sub_overlaps}.
\begin{figure}[H]
	\centering
	\includegraphics[width=10cm]{grafiki/sub_overlaps}
	\captionsetup{justification=centering}
	\caption{Wyniki zakladkowania przy zmiennej dlugosci zakladek.}
	\label{rys:sub_overlaps}
\end{figure}
	\chapter{Synteza dźwięku - modulacja częstotliwości}\label{chapter_fm}
Modulacja częstotliwości jest powszechnie kojarzona z radiem analogowym. W tym zastosowaniu, w chwilowej częstotliwości sygnału sinusoidalnego zaszyta jest informacja, która ma drogą radiową dotrzeć do odbiorcy. W przypadku syntezy dźwięku, modulowane są dowolne sygnały okresowe. Nie niosą one żadnej informacji, a dewiacje częstotliwości chwilowej wprowadza się w celu uzyskania efektów dźwiękowych.
\section{Zasada działania modulacji częstotliwości}
Modulacja częstotliwości polega na zmienianiu częstotliwości chwilowej przebiegu okresowego. Sygnał, którego częstotliwość ulega tym zmianom, nazywany jest sygnałem modulowanym czy też nośnym. Zmiany wprowadzane w częstotliwości sygnału modulowanego, to odchylenia od jego własnej częstotliwości. Mogą one być dokonywane za pomocą innego sygnału okresowego, który nazywany jest sygnałem modulującym. 
\subsection{Modulacja FM w dziedzinie czasu}
Przykładem sygnału modulującego może być sinusoida. Przebieg zmodulowany częstotliwościowo, z pominięciem fazy sygnału modulującego \cite{oland}, opisuje wyrażenie:
\begin{equation} \label{equ:fm_wzor1}
S(t)= sin(2 \pi f_c t + \beta sin(2 \pi f_m t))
\end{equation}
\begin{tabular}{ l l l l}
	gdzie: & $t$ &  - & czas w sekundach, \\
	&	$f_c$ & - &  częstotliwość sygnału nośnego,\\
	&	$f_m$ & - &  częstotliwość sygnału modulującego.\\
	&	$\beta$ & - & amplituda sygnału modulującego.\\
\end{tabular} \\ \\
Na rysunku \ref{rys:fm_wykres1} zobrazowano modulację przeprowadzoną według (\ref{equ:fm_wzor1}) z parametrami: $\beta = 15, f_c = 1200, f_m = 55$.
\begin{figure}[H]
	\centering
	\includegraphics[width=12cm]{grafiki/fm_wykres1}
	\captionsetup{justification=centering}
	\caption{Prosty przykład sygnału zmodulowanego.}
	\label{rys:fm_wykres1}
\end{figure}
Na rysunku \ref{rys:fm_arg} przedstawiono przebiegi argumentów funkcji sinus z (\ref{equ:fm_wzor1}) dla pierwszych 50 milisekund sygnału.
\begin{figure}[H]
	\centering
	\includegraphics[width=10cm]{grafiki/fm_arg}
	\captionsetup{justification=centering}
	\caption{Przebiegi czasowe argumentu funkcji sinus przed i po modulacji FM.}
	\label{rys:fm_arg}
\end{figure}
Linią ciągłą narysowano przebieg argumentu sygnału nośnego przed modulacją. Natomiast linia przerywana obrazuje przebieg argumentu sygnału po modulacji częstotliwości.
\subsection{Modulacja FM w dziedzinie częstotliwości}
Wartość $\beta$ jest także określana mianem indeksu modulacji \cite{chowning}. Widmo sygnału poddanego modulacji częstotliwości składa się z prążka środkowego na częstotliwości nośnej oraz prążków bocznych. Prążki boczne są rozmieszczone symetrycznie względem częstotliwości nośnej i leżą na częstotliwościach $f_c \pm kf_m$, gdzie $k$ jest liczbą całkowitą. Moduł widma sygnału z powyższego przykładu przestawiony został na rysunku \ref{rys:fm_widmo}. Sygnał poddany transformacji Fouriera miał długość 96000 próbek przy $F_s = 96000$, zatem indeks częstotliwości jest równoważny częstotliwości. Ze względu na sposób indeksowania w Matlabie, zaczynający się od liczby 1 (a nie od 0), w celu uzyskania wartości częstotliwości należy odjąć jedynkę od indeksu.

\begin{figure}[H]
	\centering
	\includegraphics[width=10cm]{grafiki/fm_widmo}
	\captionsetup{justification=centering}
	\caption{Moduł widma sygnału zmodulowanego (wartość indeksu 1201 należy pomniejszyć o 1 ze względu na sposób indeksowania w środowisku Matlab).}
	\label{rys:fm_widmo}
\end{figure}
Wraz ze wzrostem wartości $\beta$, wzrasta liczba znaczących prążków bocznych oraz poszerza się pasmo częstotliwościowe sygnału zmodulowanego.
Wartości poszczególnych prążków odpowiadają wartościom funkcji Bessela typu pierwszego $J_n(x)$, gdzie $n$ jest rzędem. Wykresy tych funkcji dla trzech pierwszych rzędów przedstawione są na rysunku \ref{rys:fm_bessel}.
\begin{figure}[H]
	\centering
	\includegraphics[width=10cm]{grafiki/fm_bessel}
	\captionsetup{justification=centering}
	\caption{Funkcje Bessela typu pierwszego.}
	\label{rys:fm_bessel}
\end{figure}

Moduł funkcji Bessela rzędu 0 w punkcie $\beta$ jest równy modułowi prążka na częstotliwości $f_c$. Moduł pierwszych prążków bocznych, tj. na częstotliwościach $f_c \pm f_m$ jest równy funkcji Bessela rzędu 1 w punkcie $\beta$. Postępując analogicznie można wyznaczyć moduły wszystkich znaczących prążków.

\subsection{Rozbudowana modulacja FM}
Zastosowanie modulacji FM w syntezie dźwięku ma doprowadzić do uzyskania wrażeń słuchowych. Zatem nic nie stoi na przeszkodzie, aby tę modulację rozbudowywać wedle własnego uznania. Można na przykład poddać modulacji przebieg modulujący. Niech sygnał modulujący (SM1) będzie modulowany przez SM2 (sygnał modulujący 2). Na rysunku \ref{rys:fm_modmod} przedstawiono wykresy sygnału nośnego, SM1, SM2 oraz sygnału końcowego, który jest opisany przez wzór:
\begin{equation} \label{equ:fm_modmod}
S(t)= sin(2 \pi f_c t + \beta sin(2 \pi f_m t + \gamma sin(2 \pi f_{m2} t)))
\end{equation}
\begin{tabular}{ l l l l}
	gdzie: & $t$ &  - & czas w sekundach, \\
	&	$f_c$ & - &  częstotliwość sygnału nośnego,\\
	&	$f_m$ & - &  częstotliwość SM1,\\
	&	$f_{m2}$ & - &  częstotliwość SM2,\\
	&	$\beta$ & - & amplituda SM1,\\
	&	$\gamma$ & - & amplituda SM2.\\
\end{tabular} \\ \\

\begin{figure}[H]
	\centering
	\includegraphics[width=14cm]{grafiki/fm_modmod}
	\captionsetup{justification=centering}
	\caption{Zagnieżdżona modulacja FM.}
	\label{rys:fm_modmod}
\end{figure}
Na rysunku \ref{rys:fm_arg2} przedstawiono przebiegi argumentów funkcji sinus z (\ref{equ:fm_modmod}) dla pierwszych 50 milisekund sygnału.
\begin{figure}[H]
	\centering
	\includegraphics[width=10cm]{grafiki/fm_arg2}
	\captionsetup{justification=centering}
	\caption{Przebiegi czasowe argumentu funkcji sinus przed i po zagnieżdżonej modulacji FM.}
	\label{rys:fm_arg2}
\end{figure}
Linią ciągłą narysowano przebieg argumentu sygnału nośnego przed modulacją. Natomiast linia przerywana obrazuje przebieg argumentu sygnału po zagnieżdżonej modulacji częstotliwości. Na drodze eksperymentów można uzyskiwać w ten sposób naprawdę ciekawe brzmienia.
\subsection{Uzyskiwanie konkretnych brzmień}
W literaturze można znaleźć parametry syntezy FM, które pozwalają uzyskać brzmienie zbliżone do naturalnych dźwięków. W \cite{chowning} odnaleźć można na przykład parametry, za pomocą których można uzyskać brzmienie dzwona. Parametry te wyglądają następująco:

\begin{table}[h!]
\centering
	\begin{tabular}{ |c| c| }
	\hline
	Parametr & Wartość \\
	\hline
	$f_c$ & 200 Hz \\
	\hline
	$f_m$ & 280 Hz\\
	\hline
	$\beta$ & 10\\
	\hline
	
	\end{tabular}
\captionsetup{justification=centering}
\label{tab:fm_bell}
\caption{Parametry brzmienia dzwona.}
\end{table}
Jednak dźwięk samego zmodulowanego przebiegu nie ma charakteru brzmienia dzwona. Ważnym jest zastosowanie odpowiedniej obwiedni amplitudy uzyskanego przbiegu. W tym przypadku dobry wynik daje zastosowanie funkcji zanikającej ekspotencjalnie. Przykładem takiej funkcji jest wyrażenie:
\begin{equation} \label{equ:fm_belladsr}
A(t) = e^{-t}.
\end{equation}
Przebieg obwiedni opisanej za pomocą (\ref{equ:fm_belladsr}) pokazano na rysunku \ref{rys:fm_belladsr}.
\begin{figure}[H]
	\centering
	\includegraphics[width=10cm]{grafiki/fm_belladsr}
	\captionsetup{justification=centering}
	\caption{Kształt obwiedni w syntezie dźwięku dzwona.}
	\label{rys:fm_belladsr}
\end{figure}

\section{Implementacja syntezy FM}
W celu nienaruszania struktury pętli głównej programu wykonywanego na DSP, algortym syntezy dla metody FM został dostosowany do istniejącego kodu. Stąd generowanie dźwięku jest realizowane w sposób blokowy. Blok ma rozmiar $N = 1024$ próbek. 
\subsection{Generowanie przebiegu}
W programie wykorzystywane są dwie tablice o $N$ próbkach, które naprzemiennie wysyłane są do przetwornika cyfrowo-analogowego. W czasie, gdy jedna z nich jest przetwarzana na sygnał analogowy, druga jest wypełniana nowymi próbkami przebiegu. W celu zachowania ciągłości pomiędzy blokami, wprowadzony został licznik bloków $k$, który odpowiada za odpowiednie przesuniecię w fazie generowanych sygnałów. Za generowanie pojedynczego tonu zmodulowanego częstotliwościowo odpowiada kod:
\begin{equation} \label{equ:fm_wzor2}
\text{waveform[i]} = \text{sinf}(2\pi f_c(i+kN)\frac{1}{F_s} + \beta \text{sinf}(2 \pi f_m(i+kN)\frac{1}{F_s}))
\end{equation}
\begin{tabular}{ l l l l}
	gdzie: & waveform &  - & tablica zmiennych typu float, \\
	&	$F_s$ & - & częstotliwość próbkowania,\\
	&	$i$ & - &  licznik iteracji, $i$ = 0, 1, 2, ..., N-1,\\
	&	$k$ & - &  licznik bloków.\\
\end{tabular} \\ \\
W (\ref{equ:fm_wzor2}) zastosowano funkcję "sinf" zamiast zwykłego "sin". Obie te funkcję należą do biblioteki math.h w języku C. Wybór "sinf" wynika z faktu, że funkcja "sin" operuje na zmiennych typu double (podwójna precyzja), a "sinf" na zmiennych typu float. Działania na zmiennych typu float są wykonywane szybciej. Kosztem jest mniejsza precyzja wykonywanych obliczeń.

Po wypełnieniu próbkami danej tablicy waveform, jest ona wysyłana do DAC. Licznik bloków jest modyfikowany w następujący sposób:
\begin{equation} \label{equ:fm_wzor3}
k \gets k + N - m.
\end{equation}
\begin{tabular}{ l l l l}
	gdzie: & $m$  &  - & liczba zakładkowanych próbek. \\
\end{tabular} \\ \\
Mechanizm zakładkowania, opisany w metodzie subtraktywnej, nie jest wyłączany przy aktywacji innych metod syntezy. Z tego powodu, licznik bloków jest pomniejszany o wartość $m$.
\subsection{Polifonia w syntezie FM}
Do generowania wielu tonów jednocześnie, wykorzystywana jest tablica wciśniętych klawiszy instrumentu, opisana w REALIZACJA->POLIFONIA. Jest ona przeglądana w pętli głównej programu i dla każdej częstotliwości, która się w niej znajduje, generowane jest $N$ próbek przebiegu w jednej z tablic "waveform". Generowanie pojedynczego bloku z uwzględnieniem polfionii opisuje wyrażenie:
\begin{equation} \label{equ:fm_wzor4}
\text{waveform[i]} =\left \{\begin{array}{ r l }
\text{sinf}(2\pi f_j(i+kN)\frac{1}{F_s} + \beta \text{sinf}(2 \pi f_m(i+kN)\frac{1}{F_s})), & \quad \text{dla } j = 0\\
\text{waveform[i]} + \text{sinf}(2\pi f_j(i+kN)\frac{1}{F_s} + \beta \text{sinf}(2 \pi f_m(i+kN)\frac{1}{F_s})), & \quad  \text{dla } j = 1, 2, ..., J-1
\end{array}
\right.
\end{equation}
\begin{tabular}{ l l l l}
	gdzie: & $J$ &  - & liczba wciśniętych klawiszy instrumentu, \\
		&	$i$ & - & licznik iteracji, $i$ = 0, 1, 2, ..., N-1,\\
		&	$f_j$ & - & częstotliwość j-tego tonu.\\
\end{tabular} \\ \\

Zastosowane rozwiązanie pozwala wygenerować do 6 tonów jednocześnie. 

\section{Interfejs użytkownika}
W projekcie autorskim, w przypadku syntezy FM, użytkownik może poprzez interfejs użytkownika określać parametry modulacji: amplitudę sygnału modulującego oraz jego częstotliwość.
\begin{figure}[H]
	\centering
	\includegraphics[width=12cm]{grafiki/sub_interface}
	\captionsetup{justification=centering}
	\caption{Interfejs użytkownika dla modulacji częstotliwości.}
	\label{rys:fm_interface}
\end{figure}
Na rysunku \ref{rys:fm_interface} zaznaczono dwie sekcje wykonanego interfejsu dla syntezy FM. W pierwszej z nich znajduje się przycisk, który służy do aktywacji tej metody syntezy dźwięku. Po naciśnięciu tego klawisza, DSP otrzymuje komunikat, po którym przechodzi w tryb syntezy FM.

W drugiej sekcji znajdują się suwaki pozwalające na wybór amplitudy oraz częstotliwości sygnału modulującego. Zmian tych wielkości można dokonywać na dwa sposoby:
\begin{itemize}
	\item za pomocą suwaków. W tym wypadku wartość ustawiona za pomocą suwaka zostanie automatycznie wpisana do pola tekstowego skojarzonego z tym suwakiem.
	\item Poprzez precyzyjne wpisanie wartości do pola tekstowego obok suwaka. W tej sytuacji suwak automatycznie się przesunie na pozycję odpowiadającą wpisanej wartości.
\end{itemize}
Zmianę parametrów należy zatwierdzić klawiszem "Set". Po jego naciśnięciu, odpowiednie komunikaty zostają przesłane do DSP.

	\chapter{Addytywna synteza dźwięku}\label{chapter_additive}
Synteza addytywna pozwala na utworzenie barwy dźwięku poprzez zbudowanie pełnego widma częstotliwościowego za pomocą wybranych składowych. Słowo "addytywna" odnosi się do sumowania wielu przebiegów w jeden złożony sygnał dźwiękowy. Historycznie metoda ta jest drugą najstarszą metodą syntezy, zaraz po subtraktywnej, opisanej w rozdziale \ref{chapter_subtractive}.
Zazwyczaj synteza addytywna polega na dodawaniu sygnałów reprezentujących wiele fal sinusoidalnych o różnych częstotliwościach oraz amplitudach. Metoda ta daje użytkownikowi wiele różnorodnych możliwości. Dodawanymi składowywmi sygnału nie muszą być jedynie fale sinusoidalne.
Synteza addytywna uznawana jest za odwrotność syntezy subtraktywnej. Znajduje zastosowanie w~syntezie dźwięku oraz mowy.

W niniejszym rozdziale przedstawiono zasadę działania syntezy addytywnej. Zaprezentowano wzory matematyczne, za pomocą których można uzyskać zaprojektowane brzmienie. Przedstawiono również kilka metod implementacji tego rodzaju syntezy. Na końcu rozdziału omówiono autorski interfejs użytkownika oraz wyniki realizacji syntezy addytywnej na procesorze DSP.

\section{Zasada działania syntezy addytywnej}
Synteza addytywna jest zbliżona pod niektórymi względami do analizy częstotliwościowej Fouriera. Z tego powodu jest ona zaliczana do widmowych metod syntezy. Jej postać matematyczną można jednak przedstawić również w dziedzinie czasu. Zależnie od doboru składowych dźwięku, postać syntezy addytywnej może mieć różne reprezentacje \cite{add_defins}.
%https://ccrma.stanford.edu/~jos/sasp/Additive_Synthesis_Early_Sinusoidal.html

\subsection{Postać harmoniczna} \label{pos_harm}
Dźwięk powstały w wyniku użycia syntezy addytywnej w formie harmonicznej można zapisać za pomocą wzoru:
%https://en.wikipedia.org/wiki/Additive_synthesis#:~:text=Additive%20synthesis%20is%20a%20sound,or%20inharmonic%20partials%20or%20overtones.
\begin{equation} \label{equ:addit_time_harm}
y(t) = \sum_{k=1}^{K} a_{k}\text{sin}(2\pi kf_{0}t + \phi_{k})  \\  
\end{equation}
\begin{tabular}{ l l l l}
	gdzie: & $y(t)$ &  - & sygnał wyjściowy addytywnej syntezy dźwięku, \\
	&	$k$ & - &  numer składowej harmonicznej sygnału, \\
	&	$K$ & - &  całkowita liczba składowych harmonicznych dźwięku,\\
	&	$f_{0}$ & - &  częstotliwość pierwszej składowej harmonicznej,\\
	&	$a_{k}$ & - &  amplituda składowej harmonicznej k, \\
	&	$\phi_{k}$ & - &  faza składowej harmonicznej k. \\
\end{tabular} \\

Każda składowa dźwięku uzyskanego z takiej postaci syntezy addytywnej jest wielokrotnością częstotliwości podstawowej $f_{0}$. Wykorzystanie takiej postaci pozwala przykładowo na wygenerowanie dźwięku organów. Jest to najbardziej podstawowy rodzaj syntezy addytywnej.

%https://en.wikibooks.org/wiki/Sound_Synthesis_Theory/Additive_Synthesis <<---- SCHEMAT 

Postać harmoniczna syntezy addytywnej pozwala również na uzyskanie podstawowych przebiegów używanych w syntezie subtraktywnej. Każdy z nich można wygenerować za pomocą sumowania odpowiednich składowych harmonicznych z odpowiednimi amplitudami. Przykłady takich przebiegów w formie ciągłoczasowej przedstawiono we wzorach (\ref{equ:addit_sqr}), (\ref{equ:addit_trng}) oraz (\ref{equ:addit_sawth}).

%https://en.wikipedia.org/wiki/Square_wave
\begin{equation} \label{equ:addit_sqr}
y_{Square}(t) = \sum_{k=1}^{K} \frac{1}{2k-1} \text{sin}(2\pi (2k-1)f_{0}t), \\
\end{equation}

%https://en.wikipedia.org/wiki/Triangle_wave
\begin{equation} \label{equ:addit_trng}
y_{Triangle}(t) = \sum_{k=1}^{K} \frac{(-1)^k}{(2k-1)^2} \text{sin}(2\pi (2k-1)f_{0}t),  \\
\end{equation}

%https://en.wikipedia.org/wiki/Sawtooth_wave
\begin{equation} \label{equ:addit_sawth}
y_{Sawtooth}(t) = \sum_{k=1}^{K} \frac{(-1)^k}{k} \text{sin}(2\pi kf_{0}t). \\
\end{equation}
% O tym że na tym polegały organy Hammonda

Na rysunku \ref{rys:add_sawtooth} przedstawiono wygenerowany przebieg piłokształtny, jako przykład addytywnej syntezy dźwięku w postaci harmonicznej. Na pierwszym z wykresów przedstawiono przebieg uzyskany z dwóch składowych harmonicznych, a na kolejnym z dziesięciu. Na ostatnim wykresie widać przebieg składający się z 50 SH, który jest dobrą aproksymacją idealnego przebiegu piłokształtnego. Sygnał został wygenerowany na podstawie wzoru (\ref{equ:addit_sawth}).

\begin{figure}[H]
	\centering
	\includegraphics[width=12cm]{grafiki/add_sawtooth}
	\captionsetup{justification=centering}
	\caption{Generacja przebiegu piłokształtnego za pomocą syntezy addytywnej.}
	\label{rys:add_sawtooth}
\end{figure}

\subsection{Postać nieharmoniczna} \label{pos_nieharm}
Dobieranie składowych dźwięku w syntezie addytywnej nie musi zależeć od aktualnie wybranej częstotliwości podstawowej. Niektóre instrumenty wydają dźwięk składający się ze składowych harmonicznych oraz nieharmonicznych (czyli takich, które nie są całkowitą wielokrotnością pewnej częstotliwości $f_{0}$). Zsyntezowany dźwięk w takiej postaci można opisać wzorem:
\begin{equation} \label{equ:addit_time_nieharm}
y(t) = \sum_{k=1}^{K} a_{k}\text{sin}(2\pi f_{k}t + \phi_{k})  \\  
\end{equation}
\begin{tabular}{ l l l l}
	gdzie: 	&	$f_{k}$ & - &  częstotliwość składowej sygnału k. \\
\end{tabular} \\

Postać (\ref{equ:addit_time_nieharm}) można traktować jako uogólnioną formę wzoru (\ref{equ:addit_time_harm}). Częstotliwość $f_k$ to taka, która zazwyczaj nie jest całkowitą wielokrotnością częstotliwości $f_0$. Dźwięk opisany powyższym wzorem jest generowany przez instrumenty takie jak dzwony lub perkusjonalia.

\subsection{Składowe zmienne w czasie}
%https://books.google.pl/books/about/The_Computer_Music_Tutorial.html?id=nZ-TetwzVcIC&printsec=frontcover&source=kp_read_button&redir_esc=y#v=onepage&q&f=false  , strona 140
Wzory matematyczne (\ref{equ:addit_time_harm}) i (\ref{equ:addit_time_nieharm}) pozwalają na uzyskanie jedynie stanu ustalonego zsyntezowanego brzmienia. Powtarzany jest jeden okres sygnału, co daje wrażenie słuchaczowi stałości parametrów powtarzanego dźwięku.

Składowe sygnału audio mogą jednak zmieniać się w czasie \cite{add_time_varying}. Zmiany te mogą dotyczyć ich amplitudy, jak i częstotliwości.
Taką postać syntezy addytywnej zapisuje się jako:
\begin{equation} \label{equ:addit_time_zmienne}
y(t) = \sum_{k=1}^{K} a_{k}(t)\text{sin}(2\pi f_{k}(t)t + \phi_{k})  \\  
\end{equation}
\begin{tabular}{ l l l l}
	gdzie: & $a_{k}(t)$ &  - & zmienna w czasie amplituda składowej k, \\
	&	$f_{k}(t)$ & - &  zmienna w czasie częstotliwość składowej k. \\
\end{tabular} \\

Realizacja brzmienia na podstawie wzoru (\ref{equ:addit_time_zmienne}) może być rozumiana jako zmiana parametrów instrumentu w trakcie generowania przez niego dźwięku.

\subsection{Szum w syntezie addytywnej}
%https://ccrma.stanford.edu/~jos/sasp/S_N_Synthesis.html
Tworzenie dźwięku zsyntezowanego metodą addytywną za pomocą wzorów przedstawionych powyżej jest deterministyczne. Do takiego sygnału może zostać dodana część stochastyczna. Uzyskuje się to przez wykorzystanie szumu białego, którego widmo kształtowane jest za pomocą filtru FIR o zmiennych w czasie parametrach \cite{add_szum}. Takie działanie zostało zaprezentowane we wzorze (\ref{equ:addit_szum}).

\begin{equation} \label{equ:addit_szum}
B(\omega) = F(\omega, t)e^{j\phi(\omega_{k})} \\  
\end{equation}
% Wprowadzic (t)
\begin{tabular}{ l l l l}
	gdzie: & $\phi(\omega_{k})$ &  - & faza losowa o rozkładzie równomiernym od -$\pi$ do $\pi$, \\
	& $F(\omega, t)$ &  - & obwiednia widmowa filtra FIR o zmiennych parametrach w czasie, \\
	% TU DOPISAC B(w) wyjasnienie
	&	$B(\omega)$ & - & widmo powstałej części stochastycznej. \\
\end{tabular} \\

%Synteza części stochastycznej polega na przetwarzaniu białego szumu (bazującego na losowej fazie) przez filtr FIR.
% Zajrzec do wykladu Niedzwieckiego
Po wykonaniu zależności (\ref{equ:addit_szum}), część stochastyczna zostaje przetransformowana do dziedziny czasu. Ostatnim krokiem jest zsumowanie ze sobą obu części.

Dodanie szumu do sygnału deterministycznego w syntezie addytywnej pozwala na uzyskanie dźwięków na przykład instrumentów dętych. Część stochastyczna sygnału sprawia wrażenie pobudzenia rzeczywistego stroika instrumentu strumieniem powietrza.

% Mozna jeszcze to użyć:
%https://ccrma.stanford.edu/~jos/sasp/Sines_Noise_Analysis.html

\section{Metody implementacji syntezy addytywnej}
Istnieją różne metody implementacji addytywnej syntezy dźwięku \cite{add_imp_meth}. Oznacza to, iż ten sam dźwięk może zostać uzyskany różnymi sposobami. Popularne metody realizacji syntezy addytywnej to:
\begin{itemize}
	\item bank oscylatorów,
	\item synteza wavetable,
	\item synteza IFFT.
	% https://ieeexplore.ieee.org/document/4412805
\end{itemize}
W niniejszym podrozdziale przedstawiono ograniczenia syntezy addytywnej oraz omówiono wymienione wyżej metody implementacji.

\subsection{Ograniczenia syntezy addytywnej} \label{addit_ograniczenia}
%https://en.wikipedia.org/wiki/Additive_synthesis#History
Synteza addytywna w instrumentach klawiszowych jest używana między innymi do generacji dźwięku organów, których barwa upraszczana jest zazwyczaj do kilku SH. W przypadku próby uzyskania dźwięku o większej liczbie składowych harmonicznych, złożoność obliczeniowa metody wzrasta \cite{add_ograniczenia}.
%https://ccrma.stanford.edu/~jos/pasp/Additive_Synthesis.html
Przykładowo, dla uzyskania pojedynczego dźwięku fortepianu w jakości CD-Audio,
%https://en.wikipedia.org/wiki/Compact_Disc_Digital_Audio
wymagane byłoby obliczenie około 400 fal sinusoidalnych na każdą próbkę zsyntezowanego dźwięku. Oznacza to, iż dla polifonicznej klawiatury takiego pianina mogłyby być to tysiące składowych przypadających na jedną próbkę. Z uwagi na zbyt dużą złożoność obliczeniową. obecne ograniczenia sprzętowe nie umożliwiają realizacji w czasie rzeczywistym tak skomplikowanych brzmień, za pomocą syntezy addytywnej. Powstrzymują one szybki rozwój tej metody.

\subsection{Bank oscylatorów}
%analogowe hammondy i zwykłe, ale ze są wolne
Synteza za pomocą banku oscylatorów jest najstarszą metodą implementacji addytywnej syntezy dźwięku. Opiera się ona bezpośrednio na równaniu (\ref{equ:addit_time_harm}).
Ta metoda implementacji realizowana była nawet na instrumentach analogowych. Instrumenty takie jak organy Hammonda miały kilka wirujących dysków (koła tonacyjne) posiadających w wielu miejscach nacięcia o specyficznych kształtach. Obracające tarcze powodowały zmiany w polu magnetycznym, co generowało małe napięcia w zwojnicy. Dzięki temu uzyskiwano prąd elektryczny o charakterystyce zawierającej pewne składowe harmoniczne tonu podstawowego.
% Przez te tarcze oswietlana byla fotokomorka
\begin{figure}[H]
	\centering
	\includegraphics[width=15cm]{grafiki/add_hammond_matlab}
	\captionsetup{justification=centering}
	\caption{Barwa dźwięku organów Hammonda dla różnych ilości składowych harmonicznych.}
	\label{rys:add_hammond_matlab}
\end{figure}

Implementacja na procesorach DSP, odpowiadająca wirującym dyskom, opiera się na sumowaniu wielu funkcji sin() posiadających różne częstotliwości i amplitudy. Każda taka funkcja generuje jedną składową harmoniczną syntezowanego dźwięku. Wszystkie razem traktowane są jako bank oscylatorów. Problem realizacji skomplikowanego brzmienia takim sposobem został przedstawiony w punkcie \ref{addit_ograniczenia}.

Na rysunku \ref{rys:add_hammond_matlab} przedstawiono wygenerowaną barwę organów Hammonda dla trzech różnych ustawień amplitud składowych harmonicznych. Najbardziej zauważalna różnica występuje między pierwszym i ostatnim wykresem. 
Sumowanie jedynie trzech SH pozwala uzyskać dźwięk prosty, podobny do brzmienia organów piszczałkowych. Natomiast dodanie dziewięciu składowych harmonicznych bardzo wzbogaca charakter zsyntezowanego brzmienia, które upodabnia się do dźwięków organów stosowanych w muzyce rockowej.

\subsection{Synteza wavetable} \label{add_wavetable}
%opisac tą która została zaimplementowana
Synteza wavetable (nazywana również syntezą tablicową) polega na zapisaniu jednego okresu fali dźwiękowej do tablicy typu lookup, zdefiniowanej w programie komputerowym \cite{add_wavetab_synt}. W każdej kolejnej chwili czasu działania programu, odczytywany jest odpowiedni element tablicy. Sygnał zapisany w tablicy powinien zostać kilkukrotnie nadpróbkowany (ang. over-sampled), aby móc odczytać go dla fal o niskich częstotliwościach. 
Odtwarzanie wszystkich próbek w opisanej tablicy pozwala uzyskać najniższą częstotliwość fali $f_0$. Wybieranie co drugiej próbki powoduje generowanie częstototliwości 2$f_0$, co trzeciej 3$f_0$ i tak dalej.
% Generowanie dźwięków o okreslonej czestotliwosci polega na wybieraniu 

W syntezie addytywnej użycie takiej metody implementacji może znacznie przyspieszyć generowanie kolejnych próbek składowych sygnału. Do tablicy zapisuje się jeden przebieg sinusoidalny. Odczyt jednego indeksu zabiera mniej czasu pracy procesora niż wygenerowanie próbki za pomocą funkcji sin(). Działanie takiej metody jest podobne do banku oscylatorów, ze względu na działanie w dziedzinie czasu. Różnica polega na tym, że tablicę w syntezie wavetable można traktować jako jeden oscylator, z którego należy jedynie odczytać wartości tj. znaleźć odpowiedni indeks tablicy (nie jest konieczne obliczanie na nowo próbki dla każdej składowej dźwięku, w~każdej chwili czasu).
%https://en.wikipedia.org/wiki/Lookup_table

\subsection{Synteza IFFT}
%Rysunki z matlaba
W 1990 roku P. Depalle oraz X. Rodet \cite{add_ifft_orig} przedstawili przełomową metodę implementacji syntezy addytywnej. W zaproponowanym przez nich rozwiązaniu, obliczanie składowych dźwięku nie opiera się na zestawie oscylatorów, lecz na algorytmie IFFT (ang. Inverse Fast Fourier Transform) stosowanym dla krótkookresowej analizy częstotliwościowej (STS, ang. short term spectrum) \cite{add_ifft_method}. Wyniki ich pracy dowiodły, iż w stosunku do metody banku oscylatorów udało się zmniejszyć koszt obliczeń syntezy addytywnej piętnastokrotnie (w omawianym przez autorów przypadku, gdzie obliczano 9 istotnych punktów widma w 256 próbkowym STS).

% POPRAWIĆ - DOCZYTAĆ

% polega na tym ze posuwamy sie po danych i wycinamy jakis fragment danych, a potem hanning jest i mamy W[]

% Oznaczając przez W transformatę Fouriera funckji okienkującej sygnał w[n],
% Dodanie skladowej do STS moze byc zapisane zależnoscią.

Konstrukcja pojedynczej ramki STS została przedstawiona poniżej. Niech $f_{j}$, $a_{j}$ oraz $\phi_{j}$ będą wartościami częstotliwości, amplitudy i fazy ramki czasowej składowej $j$ pożądanego sygnału $s[n]$. $W[k]$ niech będzie Transformatą Fouriera okna Hanninga $w[n]$. Widmo generowanego sygnału tworzone jest z pojedynczych prążków. W niektórych sytuacjach składową należy wyrazić za pomocą kilku prążków widma (centrowanie widma wokół częstotliwości prążka). Dodanie pojedynczej składowej do STS może zostać zapisane zależnością:

\begin{equation} \label{equ:addit_IFFT}
S[k] = a_{j}e^{i\phi_{j}}W[f_{j} - k] \\  
\end{equation}
\begin{tabular}{ l l l l}
	gdzie: & $S[k]$ &  - & STS syntezowanego sygnału, \\
	& $W[f_{j} - k]$ &  - & przesunięte w częstotliwości widmo okna Hanninga, \\
	& $a_{j}e^{i\phi_{j}}$ & - & zespolona amplituda. \\
\end{tabular} \\

Wzór (\ref{equ:addit_IFFT}) oznacza, iż widmo dyskretne $W[f_{j} - k]$ zostanie wycentrowane na prążku częstotliwości $f_{j}$ oraz przemnożone przez amplitudę $a_{j}e^{i\phi_{j}}$. Po takim działaniu w dziedzinie częstotliwości, widmo dyskretne $S[k]$ jest poddane algorytmowi IFFT, którego rezultatem jest sygnał $s[n]$, który odpowiada przetworzonemu fragmentowi $w[n]$. Kolejne okna czasowe zakładkowane są metodą overlap-add. Końcowo otrzymuje się pełen sygnał wygenerowanego dźwięku za pomocą syntezy IFFT.

Dodanie szumu do sygnału jest również bardziej efektywne obliczeniowo w przypadku stosowania opisywanej metody implementacji syntezy addytywnej. Szum biały może zostać dodany już na etapie tworzenia STS, a następnie poddany algorytmowi IFFT. Ostatecznie uzyskiwane jest okno sygnału zawierającego część deterministyczną, jak i stochastyczną.

\section{Interfejs użytkownika}
W autorskim projekcie, w ramach realizacji instrumentu klawiszowego z syntezą addytywną, zaimplementowano program generujący dźwięk organów Hammonda. 
%https://www.soundonsound.com/techniques/synthesizing-tonewheel-organs-part-1
Laurens Hammond, twórca tychże organów, zaprojektował je tak, aby użytkownik miał możliwość doboru amplitudy dziewięciu składowych harmonicznych generowanego dźwięku \cite{add_hammond_soundonsound}. Każda z nich była regulowana suwakiem na 8 możliwych poziomów. Suwaki oznaczały kolejne wartości amplitudy pierwszej, drugiej, trzeciej, czwartej, szóstej, ósmej, dziesiątej, dwunastej oraz szesnastej składowej harmonicznej dźwięku. Przykładowe różnice w generowanej barwie organów przedstawione zostały na rysunku \ref{rys:add_hammond_matlab}.

% Wyjasnic numery harmonicznych na rysunku
\begin{figure}[H]
	\centering
	\includegraphics[width=15cm]{grafiki/add_interface}
	\captionsetup{justification=centering}
	\caption{Interfejs użytkownika modułu syntezy addytywnej. Numery gałek odpowiadają kolejnym suwakom w organach Hammonda.}
	\label{rys:add_interface}
\end{figure}

Interfejs programu opiera się na oryginalnym instrumencie Laurensa Hammonda. Umożliwia on użytkownikowi regulację dziewięciu SH za pomocą gałek przedstawionych na rysunku \ref{rys:add_interface}. Można zauważyć, iż gałki posiadają również 8 poziomów wartości, które odpowiadają wartościom amplitud składowych harmonicznych. Po ustawieniu pożądanej barwy, użytkownik musi kliknąć na panelu w przycisk "Set", aby parametry syntezy addytywnej zostały wysłane do procesora DSP.

Gałki panelu zostały zrealizowane za pomocą obiektów klasy KnobControl. Naciśnięcie przycisku "Set" wywołuje funkcję, która odczytuje wartość elementu panelu syntezy addytywnej w formie tablicy bajtów. Następnie program wysyła odpowiedni identyfikator gałki do procesora DSP, a~po nim jej wartość odczytaną z tablicy.

\section{Realizacja organów na procesorze DSP}
Pierwszą próbą implementacji organów Hammonda w autorskim programie dla procesora DSP, było wykorzystanie metody banku oscylatorów. Rezultaty były jednak niezadowalające. Wywołanie funkcji sin() dziwięciokrotnie dla jednej próbki sygnału, oznaczało, iż w podejściu polifonicznym mogła ona zostać wywołana nawet dziewiędziesiąt razy (odpowiada to sytuacji naciśnięcia dziesięciu klawiszy na klawiaturze MIDI).
Pomimo zastosowanego mechanizmu zakładkowania, w generowanym dźwięku pojawiały się niepożądane artefakty świadczące o zbyt dużej złożoności obliczeniowej zaimplementowanego algorytmu. Należało zmienić podejście do implementacji syntezy addytywnej.

Ostatecznie w autorskim programie komputerowym na procesor DSP wykorzystano metodę implementacji z użyciem tablicy wavetable, która została omówiona w punkcie \ref{add_wavetable}. W niniejszym podrozdziale został przedstawiony sposób realizacji programu umożliwiającego syntezę dźwięku organów Hammonda oraz rezultaty takiej syntezy z wykorzystaniem układu TMS320C6727.

\subsection{Opis implementacji}
Zdefiniowana tablica lookup o nazwie sinLut[.] zawiera jeden okres fali sinusoidalnej. Wypełniona zostaje danymi na etapie inicjalizacji programu na procesorze DSP. Każda wartość tablicy zostaje odczytana za pomocą funkcji mySin(), która jako argumenty przyjmuje częstotliwość pożądanej fali sinusoidalnej oraz jej przesunięcie w fazie.
\begin{equation} \label{equ:addit_sinLut}
\text{samp} = \text{sinLut}[(kf\frac{N}{F_s})\mod{N}] \\  
\end{equation}
\begin{tabular}{ l l l l}
	gdzie: & \text{samp} &  - & wartość próbki odczytywanej z tablicy sinLut[.], \\
	& $k$ &  - & licznik odczytu odpowiedniej próbki z tablicy sinLut[.], \\
	& $f$ & - & częstotliwość sinusoidy, \\
	& $N$ & - & długość tablicy sinLut[.], \\
	& $F_s$ & - & częstotliwość próbkowania DAC. \\
\end{tabular} \\

Funkcja mySin() zwraca element z tablicy na podstawie wzoru (\ref{equ:addit_sinLut}). Do wygenerowania jednej próbki organów Hammonda sumowane jest 9 tak odczytanych wartości sinLut[.] (odpowiadających składowym harmonicznym). Każda próbka zsyntezowanego dźwięku ulega wymnożeniu z~obecną wartością amplitudy ADSR. Opisana czynność wykonywana jest tyle razy, ile elementów posiada jeden blok generowanego sygnału.

Procesor DSP otrzymuje z interfejsu użytkownika nastawy gałek regulatorów syntezy addytywnej za pomocą funkcji obsługi przerwania UART. Parametry te są zapisywane do tablicy add\_knobAmp[.]. Następnie są używane do syntezy dźwięku organów.

\subsection{Wyniki}
Wykorzystana metoda implementacji za pomocą tablicy wavetable umożliwiła jednoczesne wykorzystanie kilkunastu klawiszy instrumentu, bez pojawienia się niepożądanych artefaktów w~dźwięku. Zmiany amplitud poszczególnych składowych harmonicznych wpływają na generowane brzmienie w bardzo wyraźny sposób.

\begin{figure}[H]
	\centering
	\includegraphics[width=15cm]{grafiki/add_hammond_dsp}
	\captionsetup{justification=centering}
	\caption{Wizualizacja sygnału dźwiękowego organów Hammonda z procesora DSP dla różnych nastaw gałek.}
	\label{rys:add_hammond_dsp}
\end{figure}

Na rysunku \ref{rys:add_hammond_dsp} przedstawiono wizualizację sygnału organów dla różnych ustawień gałek regulatorów interfejsu użytkownika. Na wykresie pierwszym od góry wszystkie SH mają maksymalną amplitudę (konfiguracja "8888 88888"). Tak wygenerowany dźwięk bardzo przypomina brzmienie katedralnych organów piszczałkowych. Na kolejnym wykresie przedstawiono sygnał utworzony przy konfiguracji gałek na wartościach "8880 00000". Brzmienie dźwięku przy takim ustawieniu jest dużo bardziej delikatne niż w pierwszej konfiguracji. Na najniższym wykresie ustawienie amplitud to "0800 30200". Można zauważyć, iż wizualnie bardzo przypomina przebieg prostokątny. Brzmienie tak wygenerowanego dźwięku również jest zbliżone do dźwięku przebiegu prostokątnego.

Przedstawione barwy organów mogą być jeszcze bardziej urozmaicone po dodaniu takich efektów jak Chorus lub Vibrato. Dla uzyskania pełni brzmienia organów, często stosuje się również efekt pogłosu (ang. reverb).
	\chapter{Badanie właściwości układu} 
Lorem ipsum dolor sit amet, consectetur adipiscing elit, sed do eiusmod tempor incididunt ut labore et dolore magna aliqua. Ut enim ad minim veniam, quis nostrud exercitation ullamco laboris nisi ut aliquip ex ea commodo consequat. Duis aute irure dolor in reprehenderit in voluptate velit esse cillum dolore eu fugiat nulla pariatur. Excepteur sint occaecat cupidatat non proident, sunt in culpa qui officia deserunt mollit anim id est laborum.

\section{Przewidywana transmitancja}
Lorem ipsum dolor sit amet, consectetur adipiscing elit, sed do eiusmod tempor incididunt ut labore et dolore magna aliqua. Ut enim ad minim veniam, quis nostrud exercitation ullamco laboris nisi ut aliquip ex ea commodo consequat. Duis aute irure dolor in reprehenderit in voluptate velit esse cillum dolore eu fugiat nulla pariatur. Excepteur sint occaecat cupidatat non proident, sunt in culpa qui officia deserunt mollit anim id est laborum.
\begin{table}[H]
	\caption{Porównanie położenia biegunów modelu teoretycznego i zidentyfikowanego.}
	\label{bieguny_porownanie}
	\centering
	\begin{tabular}{|c|c|c|}
		\hline 
		Biegun & Model teoretyczny & Model zidentyfikowany 	\\
		\hline
	$\lambda_{0}$							& $0$ 				& $0$ 				\\	\hline
 	$\lambda_{1}$							& $5,940$			& $5,785$			\\	\hline
	$\lambda_{2}$							& $-6,648$			& $-7,076$			\\	\hline
	$\lambda_{3}$ 							& $-1,131$			& $-1,872$			\\ 	\hline
	\end{tabular} 
\end{table}

%	\include{chapter6}
	\include{chapter7}
	\chapter{Podsumowanie}
% Czy udalo sie osiagnac cel pracy. Czy udalo sie spelnic te myslniki co mamy we wstepie
Wszystkie zadania potrzebne do osiągnięcia celu niniejszej pracy zostały zrealizowane. Zaprojektowano i wykonano prototyp instrumentu klawiszowego wykorzystującego syntezę subtraktywną, addytywną, FM oraz modelowanie fizyczne. Wymienione metody generowania dźwięku zaimplementowano na procesorze sygnałowym TMS320C6727. Opracowano sterowanie transjentem dźwięku za pomocą klasycznej obwiedni ADSR. Kształt obwiedni widma zmieniano bezpośrednio przy użyciu metody subtraktywnej. Interfejs użytkownika został zrealizowany w formie aplikacji na komputerze PC. Zaimplementowano mechanizm komunikacji procesora z klawiaturą sterującą MIDI oraz z komputerem. 

Każda z metod syntezy dźwięku przedstawionych w tej pracy opiera się na innych regułach matematycznych. Synteza subtraktywna jest najstarszą z nich, stosowaną już na syntezatorach analogowych. Pozwala na zmianę brzmienia za pomocą filtrowania przebiegu czasowego posiadającego wiele składowych harmonicznych. Synteza addytywna jest odwrotnością metody subtraktywnej. Daję możliwość utworzenia widma częstotliwościowego poprzez sumowanie wielu składowych. Niestety jest ograniczana obecnymi możliwościami sprzętowymi procesorów. Istnieją jednak algorytmy pozwalające na usprawnienie efektywności tej metody, takie jak synteza IFFT. W pracy przedstawiono również generowanie dźwięku za pomocą modulacji jego częstotliwości. Synteza FM pozwala osiągać bardzo ciekawe efekty dźwiękowe, które są często stosowane w muzyce elektronicznej.
% Synteza subtraktywna - porownanie do normalnego instrumentu. Wypada spoko. Filtry faktycznie filtrują. Trzeba uwazac na efekt gibbsa.

% Synteza addytywna - organy Hammonda brzmia dobrze. Moznaby dorzucic jeszcze efekty dzwiekowe. Ogolnie synteza jest ograniczona przez to ze duza zlozonosc obliczeniowa. w przyszlosci moze sie uda prawdziwe nasladowac nią z szumem np

% Synteza FM -- Udalo sie zbadac algorytmy modulujace fale sinusoidalne, dzieki ktorym uzyskano ciekawe brzmienia. Udalo sie uzyskac dzwon. CD>>>> Pelka??

% Synteza matematyczna -- Porownano dwa instrumenty pod wzgledem syntezy falowodowej. Udalo sie uzyskac dzwieki w symulacji. Na procesorze jest ciezko, duza zlozonosc obliczeniowa dla skomplikowanych modeli. A skomplikowane modele potrzebne zeby brzmienie bylo mozliwe do osiagniecia instrumentow roznych. Duza precyzja -- nie dzialalo na floatach.
Synteza dźwięku poprzez modelowanie fizyczne instrumentów jest bez wątpienia najbardziej skomplikowaną z przedstawionych metod. Opisuję ona mechanikę fizyczną instrumentu oraz reakcję na pobudzenia instrumentalisty w postaci równań matematycznych lub schematów. Synteza falowodowa była przełomem dla metody modelowania fizycznego instrumentów pod względem uzyskiwanych efektów. Osiągnięto za jej pomocą bardzo zadowalające wyniki. Autorzy przedstawili również syntezę fizyczną poprzez pobudzenie modelu ARMA szumem, która okazała się być zbyt złożona obliczeniowo.
Niestety z uwagi na ograniczone możliwości obliczeniowe procesorów DSP, nie można zaimplementować bardzo skomplikowanych mechanizmów fizycznych. Metoda ta jest wciąż głównie używana w laboratoriach. Jedynie synteza falowodowa została wprowadzona na rynkowe instrumenty klawiszowe.

% TRUDNOSCI:
%	- skompilkowana architektura płyty PADK
%	- slabe wsparcie ze strony producenta - jeden z nich nie istnieje
%	- przenoszenie kodu na DSP, wymaga optymalizacji
Osiągnięcie celu pracy wymagało od autorów umiejętności połączenia teorii z praktyką. Postawiono przed nimi wiele wyzwań. Techniczny aspekt tej pracy, którym był projekt i realizacja instrumentu klawiszowego, wymagał zrozumienia architektury procesora oraz mechanizmów przyspieszających przetwarzanie sygnałów. 
W trakcie prac nad tym dyplomem, okazało się, iż firma produkująca płytę uruchomieniową PADK już nie istnieje. Oznaczało to brak ewentualnego wsparcia ze strony producenta. Autorom udało się jednak zainicjalizować sprzęt oraz wykorzystać duże możliwości obliczeniowe procesora.

Dużym wyzwaniem było również przenoszenie kodu z symulacji w środowisku Matlab na DSP. Symulacja nie wymagała działania w czasie rzeczywistym, w przeciwieństwie do programu na procesorze. Dodatkowo pojawiły się takie problemy jak niedokładności obliczeń spowodowane użyciem zmiennych pojedynczej precyzji. Użycie zmiennych podwójnej precyzji tworzyło jednak problem większej złożoności obliczeniowej. Zagadnienie optymalizacji kodu miało bardzo duże znaczenie dla realizowanego projektu. Nawet małe niedokładności w obliczeniach kolejnych próbek oznaczały pojawienie się niepożądanych artefaktów dźwiękowych.


% co nalezy zrobic do kontynuowania pracy
Obecnie instrumenty klawiszowe oferują sterowanie transjentami oraz obwiednią widma. Przeważnie posiadają one również klawiatury polifoniczne. Wciąż mogą jednak zostać rozwinięte na wiele różnych sposobów. Interfejsy użytkownika są często nieczytelne, a zsyntezowane brzmienia są dalekie od naturalnych barw instrumentów. Firmy zajmujące się rozwojem syntezatorów, powinny zwrócić szczególną uwagę na te dwa aspekty.

Kontynuacja badań na temat metod syntezy dźwięku powinna być skierowana głównie na modelowanie fizyczne instrumentów. Jako najmłodsza z metod pozostawia wiele nieodkrytych jeszcze obszarów. Utworzone modele mogą posiadać wiele parametrów, gdzie zmiana każdego z nich może tworzyć barwę dźwięku o zupełnie innym charakterze. Jednak aby osiągnięto zadowalające rezultaty implementacji tak skomplikowanych metod syntezy na DSP, najpierw należy pokonać ograniczenia sprzętowe. Optymalizacja algorytmów syntezy jest bardzo istotnym zagadnieniem w trakcie implementacji.


	\listoffigures      % spis obrazków
	\listoftables

	\bibliographystyle{plain}                       % styl bibliografi
	\begin{thebibliography}{20}                      % pocz¹tek œrodowiska
		\small              % spisy i bibliografie sk³adamy mniejszym stopniem pisma
		\bibitem{dzwiek_pwn}
		Encyklopedia PWN, \emph{https://encyklopedia.pwn.pl/haslo/dzwiek}, (data dostępu 23.07.2020 r.)
		
		\bibitem{synth_brief_intro}
		Martinez-Zorrilla D.:\emph{Synthesizers: A Brief Introduction}, Universitat Oberta de Catalunya, January 2008.

		\bibitem{metody_syntezy}
		T. Rolonen, V. Valimaki, M. Karjalainen: \emph{Evaluation of Modern Sound Synthesis Methods}, ISSN 1239-1867, Espoo, 1998 r.

		\bibitem{bilbao}
		Bilbao S.: \emph{https://ccrma.stanford.edu/~bilbao/booktop/node12.html}, online book, (data dostępu 08.08.2020 r.)
		
		\bibitem{czyzewski_dzwiek_cyfrowy}
		 Czyżewski A.:\emph{Dźwięk cyfrowy}, Akademicka Oficyna Wydawnicza EXIT, Warszawa 1998 r.
		 
 		\bibitem{misra_cook_przetw_syg}
		 Misra A., Cook P. R.: \emph{Toward synthesized environments: A Survey Of Analysis And Synthesis Methods For Sound Designers And Composers}, Princeton University, Departament of Computer Science, 2009 r.
		
		\bibitem{synthtopia}
		Synthtopia: \emph{https://www.synthtopia.com/content/2019/01/11/analog-vs-digital-synthesizer-blind-test/}, (data dostępu 09.08.2020 r.)

		\bibitem{andertons}
		Blog Andertons Music CO.: \emph{https://www.synthtopia.com/content/2019/01/11/analog-vs-digital-synthesizer-blind-test/}, (data dostępu 09.08.2020 r.)
	
		\bibitem{dokumentacja_midi}
		The MIDI Manufacturers Association, \emph{The Complete MIDI 1.0 Detailed Specification}, wersja 96.1, Los Angeles, CA, 1996 r.
		
		\bibitem{dokumentacja_PADK}
		Lyrtech, \emph{Dokumentacja Professional Audio Development Kit, Technical Reference Guide}, Wrzesień 2007.
		
		\bibitem{dokumentacja_ti6727}
		Dokumentacja procesora TMS320C6727, \emph{www.ti.com}, maj 2005.
		
		\bibitem{dokumentacja_mcasp}
	    Texas Instruments, \emph{McASP Design Guide - Tips, Tricks and Practical Examples. (SPRACK0)}, Styczeń 2019 r.
		
		\bibitem{dokumentacja_dmax}
		Texas Instruments, \emph{TMS320C672x DSP Dual Data Movement Accelerator (dMAX) - Reference Guide (SPRS268E)}, maj 2005.
		
		\bibitem{lesnicki}	% \bibitem{etykieta}
		Andrzej Le\'snicki, \emph{Technika Cyfrowego Przetwarzania Sygna\l{}\'ow }, Wydawnictwo Politechniki Gda\'nskiej, 2016.
		
		\bibitem{schafer}
		A. V. Oppenheim, R. W.Schafer, \emph{Cyfrowe przetwarzanie sygnałów}, 113-114, WYD. KOMUNIKACJI I ŁĄCZNOŚCI, 1979.
		
		\bibitem{alles}
		 H. G. Alles, \emph{Music Synthesis Using Real Time Digital Techniques}, Proc. IEEE, Vol. 68, No. 4, Apr. 1980, 439-440.

		\bibitem{chowning}
		John M. Chowning, \emph{The Synthesis of Complex Audio Spectra by Means of Frequency Modulation}, 526-534, 1973, https://web.eecs.umich.edu/~fessler/course/100/misc/chowning-73-tso.pdf, (data dostępu 25.08.2020 r.)
			
	\end{thebibliography}
	
\end{document}