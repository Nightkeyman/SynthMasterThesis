% !TeX spellcheck = pl_PL
\chapter{Realizacja instrumentu muzycznego}
Implementacja tworzonego syntezatora muzycznego realizowana jest na wspomnianej we wstępie płycie PADK, która opiera swoje działanie na procesorze DSP. W ramach tego implementowane jest wiele mechanizmów, które są konieczne do kompletnego działania projektu: przetwarzanie sygnałów z klawiatury muzycznej, wydobycie dźwięku z płyty PADK oraz komunikacja z interfejsem użytkownika. W niniejszym rozdziale przedstawiono podejście do realizacji wymienionych zadań. Implementacja poszczególnych algorytmów syntezy dźwięku przedstawiona zostanie w kolejnych rozdziałach.

Cały kod programu na procesor DSP napisany został w języku C. Program pisano w środowisku Code Composer Studio v6, które jest dedykowane do procesorów firmy Texas Instruments. Wgrywanie kodu na płytę odbywało się poprzez użycie debuggera XDS510 firmy Spectrum Digital. Debugger połączony jest z płytą PADK przez taśmę, a następnie przejściówkę 8-pinową.

% zdjęcie PADK z debuggerem

\section{Układ instrumentu}
% Zdjęcie i opis co z czym łączymy

\section{Płyta PADK}

\subsection{Procesor TMS320C6727}
%https://www.ti.com/lit/ds/symlink/tms320c6727.pdf?ts=1595978326019&ref_url=https%253A%252F%252Fwww.google.com%252F 
%--> glowne feature'y
%--> strona 13 wypisane w myslnikach np busy

\subsection{Peryferia komunikacyjne}

\subsection{Przetworniki DAC}



\section{Mechanizmy szybkiego przetwarzania danych}
Procesor sygnałowy TMS320C6727 jest przeznaczony między innymi do szybkiego przetwarzania danych i sygnałów. Poza wysoką szybkością taktowania procesora, firma Texas Instruments zawiera dodatkowe mechanizmy akceleracji przepływu danych i sygnałów. Zostały one opisane w poniższych podpunktach.

\subsection{McASP}
% https://www.ti.com/lit/an/sprack0/sprack0.pdf?ts=1595842361762&ref_url=https%253A%252F%252Fwww.google.com%252F
McASP to akronim od Multichannel Audio Serial Port. Jest to komunikacyjne urządzenie peryferyjne dedykowane do przetwarzania danych audio lub wideo. Został zaprojektowany w celu przypadków wymagających wielokanałowego przetwarzania dźwięku. Jedną z najbardziej przydatnych właściwości narzędzia McASP jest schemat wielozegarowy. Pozwala on na niezależność pomiędzy portami odbierającymi i nadającymi. Komunikacja, którą zarządza McASP może odbywać się poprzez interfejs I2S (ang. Inter-IC Sound), I2C (ang. Inter-Integrated Circuit) lub SPI (ang. Serial Peripheral Interface).

Kiedy dane przepływają przez McASP, mogą zostać dostosowane tak, aby reprezentacja stałoprzecinkowa używana przez kod aplikacji była niezależna od reprezentacji używanej przez urządzenia zewnętrzne, bez wymagania dodatkowej konwersji przez procesor.

W niniejszej pracy narzędzie McASP stosowane jest do przepływu danych między DAC a procesorem. W ramach inicjalizacji McASP ustawia odpowiednią szybkość próbkowania dla modułu DAC.

\subsection{dMAX}
%https://www.ti.com/lit/ug/spru795d/spru795d.pdf?ts=1595843361787&ref_url=https%253A%252F%252Fwww.google.com%252F, strona 14, Overview
Kontroler dMAX (Dual Data Movement Accelerator) obsługuje transfery zaprogramowane przez użytkownika pomiędzy kontrolerem pamięci wewnętrznej i urządzeniami peryferialnymi na procesorach DSP firmy TI. Mechanizm ten jest dedykowany szczególnie dla procesorów z serii C672x.

Zasada działania dMAX opiera się na sygnałach zdarzeń (ang. event signals). Zdarzenie zdefiniowane jest jako zmiana wartości logicznej odpowiadającego sygnału zdarzeń w rejestrze flag zdarzeń. Zdarzenie może być używane jako: wzbudzenie rozpoczęcia transferu danych lub spowodowanie wystąpienia przerwania dla CPU. Wszystkie zdarzenia posortowane są w dwie grupy: grupa niskiego priorytetu oraz grupa wysokiego priorytetu. Mechanizm dMAX może równolegle przetwarzać dwa żądania zdarzeń z każdej z grupy.

Częścią mechanizmu dMAX jest również bufor cyrkulacyjny FIFO. Pozwala on na równoczesny, asynchroniczny odczyt i zapis danych do jednego bufora dwustronnego. Narzędzie dMAX wykrywa kiedy dane zostają zapisane do bufora i natychmiastowo wywołuje odwrócenie go. Po odwróceniu, zapisane chwilę wcześniej dane mogą zostać odczytane z drugiej strony bufora, natomiast równocześnie kolejne dane zapisywane są po pierwszej stronie.


\section{Komunikacja z płytą PADK}
Do płyty PADK dołączone zostały dodatkowe biblioteki firmy Lyrtech, które ułatwiają obsługę komunikacji z płytą. Przykładem interfejsów, do których zostały utworzone nadmienione biblioteki to UART oraz MIDI.

\subsection{Komunikacja z klawiaturą muzyczną  (MIDI)}


\subsection{Komunikacja z interfejsem użytkownika (UART)}



\section{Wydobycie dźwięku (DAC)}
% pamiętać o: odniesienie do dMAX z buforem cyrkulacyjnym



\section{Klawiatura polifoniczna}




\section{Wykorzystanie algorytmu FFT}
% jakl linkujemy biblioteke, o bibliotece
%http://www.secs.oakland.edu/~ganesan/old/courses/CSE671SU08/CSE%20671%20Lab%204%20ANC%20Code/Lee's%20adaptive%20wiener%20filter/DSPF_sp_cfftr2_dit.h
%

W pamięci ROM procesorów z serii TMS320C672x zostały umieszczone specjalnie zoptymalizowane biblioteki, dedykowane do szybkiego przetwarzania sygnałów cyfrowych. Skompilowane biblioteki zostały napisane w asemblerze w celu większej efektywności obliczeniowej. 

%https://www.ti.com/lit/an/spraas8/spraas8.pdf?ts=1596035199216&ref_url=https%253A%252F%252Fwww.google.com%252F
W naszym projekcie, użyta została biblioteka \emph{DSP Library} z procesora TMS320C6727, która posiada efektywne obliczeniowo funkcje takie jak: transformacja FFT, odwracanie macierzy lub operacja na wektorach. W celu zawarcia jej w projekcie z programem na DSP, należało zmodyfikować komende pliku linkera dodając do niej bibliotekę o nazwie \emph{c67xdsplibR.lib}. Dodatkowo zmieniono sekcje pamięci, tak aby linker został odpowiednio poinstruowany, do których obszarów pamięci powinien się odnieść.

%http://software-dl.ti.com/jacinto7/esd/processor-sdk-rtos-jacinto7/latest/exports/docs/dsplib_c66x_3_4_0_0/docs/doxygen/html/dsplib_html/group___f_f_t.html
Do pełnej transformacji FFT użyto dwóch funkcji z nadmienionej biblioteki: DSPF\_sp\_cfftr2\_dit() oraz DSPF\_sp\_bitrev\_cplx(). Pierwsza z nich odpowiada za przeprowadzenie algorytmu FFT radix-2 dla liczb zmiennoprzecinkowych. Sygnał na wejściu powinien posiadać N liczb zespolonych ułożonych w tablicy kolejno w pary liczb rzeczywistych i urojonych. Jako drugi argument wejściowy, funkcja przyjmuje tablicę współczynników obrotu dla FFT o długości N/2. Opisywana funkcja może również zostać użyta do uzyskania transformaty odwrotnej poprzez zmianę kolejności współczynników obrotu, a na końcu podzielenie wynikowej transformaty przez N, czyli według wzoru \ref{equ:idft_upr}.



% bitreverse opisac to co mamy, twiddle factory itp

\section{Generowanie przebiegów czasowych}
% waveformy wykorzystywane w subtractive, additive oraz FM - dlatego opisywane tutaj


\section{ADSR}



\section{Interfejs użytkownika}