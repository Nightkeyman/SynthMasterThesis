% !TeX spellcheck = pl_PL
\chapter{Realizacja instrumentu muzycznego}
Implementacja tworzonego syntezatora muzycznego realizowana jest na wspomnianej we wstępie płycie PADK, która opiera swoje działanie na procesorze DSP. W ramach tego implementowane jest wiele mechanizmów, które są konieczne do kompletnego działania projektu: przetwarzanie sygnałów z klawiatury muzycznej, wydobycie dźwięku z płyty PADK oraz komunikacja z interfejsem użytkownika. W niniejszym rozdziale przedstawiono podejście do realizacji wymienionych zadań. Implementacja poszczególnych algorytmów syntezy dźwięku przedstawiona zostanie w kolejnych rozdziałach.

Cały kod programu na procesor DSP napisany został w języku C. Program pisano w środowisku Code Composer Studio v6, które jest dedykowane do procesorów firmy Texas Instruments. Wgrywanie kodu na płytę odbywało się poprzez użycie debuggera XDS510 firmy Spectrum Digital. Debugger połączony jest z płytą PADK przez taśmę, a następnie przejściówkę 8-pinową.

% zdjęcie PADK z debuggerem

\section{Parametry procesora TMS320C6727}

\section{Mechanizmy szybkiego przetwarzania danych}
Procesor sygnałowy TMS320C6727 jest przeznaczony między innymi do szybkiego przetwarzania danych i sygnałów. Poza wysoką szybkością taktowania procesora, firma Texas Instruments zawiera dodatkowe mechanizmy akceleracji przepływu danych i sygnałów. Zostały one opisane w poniższych podpunktach.

\subsection{McASP}
% https://www.ti.com/lit/an/sprack0/sprack0.pdf?ts=1595842361762&ref_url=https%253A%252F%252Fwww.google.com%252F
McASP to akronim od Multichannel Audio Serial Port. Jest to komunikacyjne urządzenie peryferyjne dedykowane do przetwarzania danych audio lub wideo. Został zaprojektowany w celu przypadków wymagających wielokanałowego przetwarzania dźwięku.

Jedną z najbardziej przydatnych rzeczy w odniesieniu do narzędzia McASP jest schemat wielozegarowy. Pozwala on na niezależność pomiędzy portami odbierającymi i nadającymi

\subsection{dMAX}
%https://www.ti.com/lit/ug/spru795d/spru795d.pdf?ts=1595843361787&ref_url=https%253A%252F%252Fwww.google.com%252F, strona 14, Overview
Kontroler dMAX (Dual Data Movement Accelerator) obsługuje transfery zaprogramowane przez użytkownika pomiędzy kontrolerem pamięci wewnętrznej i urządzeniami peryferialnymi na procesorach DSP firmy TI. Mechanizm ten jest dedykowany szczególnie dla procesorów z serii C672x.

Zasada działania dMAX opiera się na sygnałach zdarzeń (ang. event signals). Zdarzenie zdefiniowane jest jako zmiana wartości logicznej odpowiadającego sygnału zdarzeń w rejestrze flag zdarzeń. Zdarzenie może być używane jako: wzbudzenie rozpoczęcia transferu danych lub spowodowanie wystąpienia przerwania dla CPU. Wszystkie zdarzenia posortowane są w dwie grupy: grupa niskiego priorytetu oraz grupa wysokiego priorytetu. Mechanizm dMAX może równolegle przetwarzać dwa żądania zdarzeń z każdej z grupy.

Częścią mechanizmu dMAX jest również bufor cyrkulacyjny FIFO. Pozwala on na równoczesny, asynchroniczny odczyt i zapis danych do jednego bufora dwustronnego. Narzędzie dMAX wykrywa kiedy dane zostają zapisane do bufora i natychmiastowo wywołuje odwrócenie go. Po odwróceniu, zapisane chwilę wcześniej dane mogą zostać odczytane z drugiej strony bufora, natomiast równocześnie kolejne dane zapisywane są po pierwszej stronie.


\section{Komunikacja z klawiaturą muzyczną  (MIDI)}




\section{Komunikacja z interfejsem użytkownika (UART)}




\section{Interfejs użytkownika}




\section{Wydobycie dźwięku (DAC)}




\section{Klawiatura polifoniczna}




\section{Implementacja FFT}
% jakl linkujemy biblioteke, o bibliotece
% bitreverse opisac to co mamy


\section{Generowanie przebiegów czasowych}



\section{ADSR}