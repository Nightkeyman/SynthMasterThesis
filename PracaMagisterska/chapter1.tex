\chapter{Wstęp i cel pracy}
Muzyka znajduje miejsce w życiu każdego człowieka. Jest to dziedzina, w której przodkowie naszej rasy zostawili swoje odciski już w czasach prehistorycznych. Wraz z rozwojem technologicznym, ludzie zaczęli używać coraz bardziej skomplikowanych urządzeń do tworzenia muzyki. W pierwszej połowie XX wieku zaczęto używać instrumentów bazujących na układach elektronicznych. Na początku potrafiły one z siebie wydobywać niezłożone dźwięki, lecz z czasem zaczęły mieć coraz więcej możliwości. W drugiej połowie XX wieku na rynku zaczęły pojawiać się cyfrowe instrumenty klawiszowe. Technologia cyfrowego przetwarzania sygnałów otworzyła nowe wrota w dziedzinie muzyki elektronicznej.

Syntezatory analogowe miały wiele niedoskonałości: nie pozwalały one na tworzenie dokładnie pożądanych charakterystyk filtrów, miały ograniczone możliwości modelowania dźwięku oraz często się rozstrajały. Dziedzina nauki zajmująca się syntezą dźwięku zaczęła rozkwitać wraz z rozwojem procesorów sygnałowych (DSP). Pojawienie się technologii cyfrowego przetwarzania sygnałów umożliwiło tworzenie dużo bardziej skomplikowanych brzmień. Procesory sygnałowe pozwoliły manewrować obwiednią dźwięku oraz jego widmem w prawie dowolny sposób. Jednak zmiana sposobu syntezy z analogowego na cyfrowy utworzyła również nowe problemy. Pojawiły się kwestie takie jak dobieranie prędkości próbkowania DSP i odtwarzanie brzmień w czasie rzeczywistym. Ze względu na to, że na takich procesorach wykorzystywane są skomplikowane algorytmy matematyczne, należało przykładać uwagę do optymalizacji czasu wykonywania nawet najprostszych zadań.

Motywem realizacji tej pracy magisterskiej jest możliwość zbadania wielu metod syntezy dźwięku na procesorze sygnałowym. Z uwagi na to, iż imitacje instrumentów występujące na wielu syntezatorach są wciąż niezadowalające dla muzyków, istnieje perspektywa poprawy tych brzmień. Horyzont elementów, które mogą zostać udoskonalone w tej dziedzinie, jest w zasadzie tak samo szeroki jak ilość instrumentów występujących na świecie. Dzięki informacji płynącej z wielu prac naukowych prowadzonych na temat syntezy brzmień, można będzie sprawdzić implementacje zoptymalizowanych algorytmów na wysoce wydajnym procesorze sygnałowym. Dodatkowo tworzenie funkcjonalności instrumentu klawiszowego jest nieograniczone. Zaprojektowanie przejrzystego interfejsu o wielu możliwościach jest wciąż udoskonalane przez największe firmy zajmujące się instrumentami klawiszowymi. Pomimo tego, wiele syntezatorów w wysokich cenach ma wciąż nieczytelne interfejsy z wieloma problemami.

Celem tej pracy jest zrealizowanie polifonicznego instrumentu klawiszowego, który będzie miał możliwość korzystania z metod syntezy dźwięku przedstawionych w tej pracy. Realizacja projektu opierać się będzie na zestawie uruchomieniowym Professional Audio Development firmy Lyrtech procesora sygnałowego TMS320C6727 oraz klawiaturze z interfejsem USB MIDI. Jednym z celów jest również wykorzystanie dużych możliwości obliczeniowych wykorzystywanego procesora. Głównymi zadaniami do wykonania, w celu ukończenia projektu, są:

\begin{enumerate}
    \item opracowanie i implementacja na procesorze sygnałowym algorytmów generowania 	wielotonów harmonicznych i dźwięków szumowych,
    
    \item opracowanie metod sterowania barwą generowanych dźwięków dla różnych metod syntezy 	barw,
    
    \item realizacja interfejsu użytkownika.
\end{enumerate}

W rozdziale drugim niniejszej pracy przedstawiono wprowadzenie teoretyczne z dziedzin ściśle powiązanych z tą pracą. Zaprezentowana została wiedza z zakresu przetwarzania sygnałów, teorii muzyki oraz podstawy syntezy dźwięku. Przedstawiono również porównanie analogowego i cyfrowego przetwarzania sygnałów, aby lepiej uzasadnić wybór metody realizacji syntezatora dźwięku. Informację tam zamieszczone są niezbędne do pełnego zrozumienia dalszej części pracy.

W rozdziale trzecim zaprezentowano implementację instrumentu klawiszowego – sprzętową oraz programową. Przedstawienie programu oraz platformy sprzętowej pozwala na swobodną prezentację implementacji metod syntezy dźwięku w dalszych rozdziałach pracy. Zaprezentowano przebieg programu, w tym użycie algorytmu FFT w praktyce oraz przetwarzanie sygnałów z protokołu MIDI.

Rozdział czwarty przedstawia realizację interfejsu użytkownika do instrumentu klawiszowego. Umożliwia on wybieranie odpowiedniej metody tworzenia dźwięku i dobranie pożądanych parametrów. Interfejs zaimplementowano jako aplikację okienkową na komputerze PC komunikującą się z programem na procesorze sygnałowym.

Kolejne cztery rozdziały traktują o metodach syntezy dźwięków, które wykorzystywane są w instrumentach klawiszowych. Każdy z nich przedstawia na wstępie teorię z zakresu danej metody, a następnie jej realizację w praktyce. Rozdział piąty mówi o najbardziej popularnej metodzie – metodzie subtraktywnej, która jest powszechnie wykorzystywana w syntezatorach analogowych. Rozdział szósty prezentuje metodę addytywną i prezentuje możliwość utworzenia organów z jej pomocą. W rozdziale siódmym omówiono syntezę dźwięku za pomocą modulacji częstotliwości (metoda FM). Dodatkowo przedstawiono efekty jakie można nakładać na wcześniej zsyntezowane dźwięki. Rozdział ósmy przedstawia najbardziej skomplikowaną metodę syntezy – metodę modelowania matematycznego, która jednocześnie daje największe możliwości w uzyskaniu pożądanego dźwięku.

Rozdział dziewiąty jest podsumowaniem całej pracy. Zweryfikowano w nim osiągnięcie postawionego celu oraz przedstawiono wnioski płynące z całej pracy.