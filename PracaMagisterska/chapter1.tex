\chapter{Wstęp i cel pracy}
Muzyka znajduje miejsce w życiu każdego człowieka. Jest to dziedzina, która była rozwijana już od czasów prehistorycznych. Wraz z rozwojem technologicznym, ludzie zaczęli używać skomplikowanych urządzeń do tworzenia muzyki. W pierwszej połowie XX wieku zaczęto używać instrumentów bazujących na układach elektronicznych. Na początku potrafiły one wydobywać z siebie proste dźwięki. Z czasem zaczęły mieć one coraz większe możliwości. Barwa wydobytego z nich dźwięku była bardziej złożona. W drugiej połowie XX wieku na rynku zaczęły pojawiać się cyfrowe instrumenty klawiszowe. Technologia cyfrowego przetwarzania sygnałów utworzyła wiele możliwości dla twórców muzyki elektronicznej.

Syntezatory analogowe posiadały wiele niedoskonałości: nie pozwalały na tworzenie  dokładnych charakterystyk filtrów, miały ograniczone możliwości modelowania dźwięku oraz często się rozstrajały. Dziedzina nauki zajmująca się syntezą dźwięku zaczęła popularyzować się wraz z rozwojem procesorów sygnałowych (DSP). Pojawienie się technologii cyfrowego przetwarzania sygnałów umożliwiło tworzenie dużo bardziej skomplikowanych brzmień. Procesory sygnałowe pozwoliły na zmianę obwiedni dźwięku oraz jego widma w prawie dowolny sposób. Jednak zmiana sposobu syntezy z analogowego na cyfrowy utworzyła również nowe problemy. Pojawiły się kwestie takie jak dobieranie prędkości próbkowania DSP i odtwarzanie brzmień w czasie rzeczywistym. Ze względu na to, że na takich procesorach wykorzystywane są skomplikowane algorytmy matematyczne, należało zwrócić uwagę na optymalizację czasu wykonywania nawet najprostszych zadań.

Motywem realizacji tej pracy jest możliwość badania różnych metod syntezy dźwięku na procesorze sygnałowym. Imitacje dźwięków prawdziwych instrumentów występujących na wielu syntezatorach wciąż brzmią nienaturalnie. Istnieje jednak możliwość poprawy tych brzmień. Liczba elementów, które mogą zostać udoskonalone w tej dziedzinie, jest tak dużo jak liczba instrumentów klawiszowych na rynku. Dzięki informacji zawartej w wielu pracach naukowych, prowadzonych na temat syntezy brzmień, można sprawdzić implementację zoptymalizowanych algorytmów na wysoce wydajnym procesorze sygnałowym. Rozwijanie dodatkowych właściwości instrumentu klawiszowego jest nieograniczone. Zaprojektowanie przejrzystego interfejsu o wielu opcjach jest wciąż udoskonalane przez największe firmy zajmujące się instrumentami klawiszowymi. Pomimo tego, wiele drogich syntezatorów ma wciąż nieczytelne interfejsy z wieloma wadami.

Celem tej pracy jest zrealizowanie polifonicznego instrumentu klawiszowego, który będzie posiadał możliwość wykorzystania metod syntezy dźwięku przedstawionych w tej pracy. Realizacja projektu opierać się będzie na zestawie uruchomieniowym Professional Audio Development firmy Lyrtech procesora sygnałowego TMS320C6727 oraz klawiaturze z interfejsem USB MIDI. Jednym z celów jest również wykorzystanie dużych możliwości obliczeniowych wykorzystywanego procesora. Głównymi zadaniami do wykonania są:

\begin{enumerate}
    \item opracowanie i implementacja na procesorze sygnałowym algorytmów generowania 	wielotonów harmonicznych i dźwięków szumowych,
    
    \item opracowanie metod sterowania barwą generowanych dźwięków dla różnych metod syntezy 	barw,
    
    \item realizacja interfejsu użytkownika.
\end{enumerate}

W rozdziale drugim niniejszej pracy przedstawiono wprowadzenie teoretyczne z dziedzin ściśle powiązanych z tą pracą. Zaprezentowana została wiedza z zakresu przetwarzania sygnałów, teorii muzyki oraz podstawy syntezy dźwięku. Przedstawiono również porównanie analogowego i cyfrowego przetwarzania sygnałów, aby lepiej uzasadnić wybór metody realizacji syntezatora dźwięku. Informację tam zamieszczone są niezbędne do pełnego zrozumienia dalszej części pracy.

W rozdziale trzecim zaprezentowano implementację instrumentu klawiszowego – sprzętową oraz programową. Przedstawienie programu oraz platformy sprzętowej pozwala na swobodną prezentację w dalszych rozdziałach pracy implementacji metod syntezy dźwięku. Zaprezentowano przebieg programu, w tym użycie algorytmu FFT na procesorze oraz przetwarzanie sygnałów z protokołu MIDI.

Rozdział czwarty opisuje realizację interfejsu instrumentu klawiszowego. Umożliwia on wybieranie odpowiedniej metody tworzenia dźwięku oraz dobranie parametrów syntezy. Interfejs zaimplementowano jako aplikację okienkową na komputerze PC komunikującą się z programem na procesorze sygnałowym.

Kolejne cztery rozdziały opisują metody syntezy dźwięków, które wykorzystywane są w instrumentach klawiszowych. Każdy z nich przedstawia teorię z zakresu danej metody, a następnie jej realizację w praktyce. Rozdział piąty mówi o najbardziej popularnej metodzie – metodzie subtraktywnej, która jest powszechnie wykorzystywana w syntezatorach analogowych. Rozdział szósty prezentuje metodę addytywną i możliwość utworzenia za jej pomocą dźwięku organów. W rozdziale siódmym omówiono syntezę dźwięku z użyciem modulacji częstotliwości (metoda FM). Dodatkowo przedstawiono efekty jakie można nałożyć na wcześniej zsyntezowane dźwięki. Rozdział ósmy przedstawia najbardziej skomplikowaną metodę syntezy – metodę modelowania matematycznego, która jednocześnie daje największe możliwości uzyskania pożądanego dźwięku.

Rozdział dziewiąty jest podsumowaniem całej pracy. Zweryfikowano w nim osiągnięcie postawionego celu oraz przedstawiono wnioski płynące z całej pracy.