\chapter{Wstęp i cel pracy}
Muzyka jest ważnym aspektem życia każdego człowieka. Jest to dziedzina, która była rozwijana już od czasów prehistorycznych. %Napisac o instrumentach klasycznych. Instrumenty używały drgające struny, membrany , idiofony (marimba), Podac przyklady
%https://mlodytechnik.pl/eksperymenty-i-zadania-szkolne/wynalazczosc/29983-historia-wynalazkow-instrumenty-muzyczne
Na początku ludzie wykorzystywali proste instrumenty rytmiczne takie jak bębny czy grzechotki, które służyły do odprawiania rytuałów plemiennych. W starożytności wynaleziono instrumenty dęte oraz strunowe. W okresie średniowiecza i renesansu powstały instrumenty smyczkowe, chordofony (na przykład klawesyn) oraz aerofony (na przykład organy).
Wraz z rozwojem technologicznym, ludzie zaczęli używać skomplikowanych urządzeń do tworzenia muzyki. W pierwszej połowie XX wieku zaczęto używać instrumentów bazujących na układach scalonych. \cite{historia_instr}
% ogarnac po krótce budowe analogowych jak sie rozwijała
Na początku potrafiły one wydobywać z siebie proste dźwięki. Z czasem zaczęły posiadać coraz większe możliwości brzmieniowe. W drugiej połowie XX wieku cyfrowe instrumenty klawiszowe zaczęły być dostępne dla szerokiego kręgu odbiorców. Technologia cyfrowego przetwarzania sygnałów stworzyła nowe możliwości dla kompozytorów muzyki elektronicznej.

Syntezatory analogowe posiadały wiele niedoskonałości: nie pozwalały na precyzyjne odtworzenie pożądanych charakterystyk filtrów, miały ograniczone możliwości modelowania dźwięku oraz często z powodu starzenia się elementów elektronicznych zmieniały swoje parametry z upływem czasu. Dziedzina nauki zajmująca się syntezą dźwięku zaczęła dynamicznie rozwijać się wraz z rozwojem procesorów sygnałowych (DSP). Pojawienie się technologii cyfrowego przetwarzania sygnałów umożliwiło tworzenie dużo bardziej skomplikowanych brzmień. Procesory sygnałowe pozwalały na praktycznie nieograniczone zmiany charakterystyk generowanych dźwięków. Jednak zmiana sposobu syntezy z analogowej na cyfrową spowodowała pojawienie się również nowe ograniczenia. W systemach cyfrowych pojawił się problem doboru próbkowania sygnałów wejściowych wpływający na szerokość pasma ich przetwarzania.
Ważnym ograniczeniem była konieczność optymalizacji czasu wykonywania skomplikowanych algorytmów, wynikająca z konieczności ich zakończenia, do momentu pobrania kolejnej próbki.

Motywem realizacji tej pracy jest możliwość badania różnych metod syntezy dźwięku na procesorze sygnałowym. 
Autorzy zwrócili uwagę, iż w dotychczas spotykanych syntezatorach, próby imitacji brzmień naturalnych instrumentów, są dalekie od doskonałości.

Celem tej pracy było zrealizowanie polifonicznego instrumentu klawiszowego, który będzie wykorzystywał kilka typowych metod syntezy dźwięku. Realizacja projektu opierała na zestawie uruchomieniowym Professional Audio Development firmy Lyrtech procesora sygnałowego TMS320C6727 oraz klawiaturze z interfejsem USB MIDI. Jednym z celów było również wykorzystanie dużych możliwości obliczeniowych zastosowanego procesora. Głównymi zadaniami do wykonania były:

\begin{enumerate}
    \item opracowanie i implementacja na procesorze sygnałowym algorytmów generowania 	wielotonów harmonicznych i dźwięków szumowych,
    
    \item opracowanie metod sterowania barwą generowanych dźwięków dla różnych metod syntezy barw,
    
    \item realizacja interfejsu użytkownika.
\end{enumerate}
W niniejszej pracy omówiono realizację powyższych celów.

W rozdziale drugim przedstawiono informacje teoretyczne z zakresu teorii muzyki, podstaw syntezy dźwięku oraz cyfrowego przetwarzania sygnałów. W celu uzasadnienia wyboru cyfrowej syntezy dźwięku, przedstawiono porównanie wad i zalet metod analogowych i cyfrowych przetwarzania sygnałów. Informację tam zamieszczone są niezbędne do pełnego zrozumienia dalszej części pracy.

W rozdziale trzecim zaprezentowano implementację autorskiego instrumentu klawiszowego – sprzętową oraz programową. Opis działania programu oraz własności platformy sprzętowej pozwala na swobodną prezentację w dalszych rozdziałach pracy implementacji metod syntezy dźwięku. Zaprezentowano poszczególne moduły programu, w tym algorytm FFT na procesorze oraz metody sterowania protokołu MIDI. W tym samym rozdziale autorzy opisali realizację interfejsu instrumentu klawiszowego. Umożliwia on wybieranie odpowiedniej metody tworzenia dźwięku oraz dobranie parametrów ich syntezy. Interfejs zaimplementowano jako aplikację okienkową na komputerze PC, komunikującą się z programem działającym na procesorze sygnałowym.

W kolejnych czterech rozdziałach opisano metody syntezy dźwięków, które wykorzystywane są w instrumentach klawiszowych. Każdy z nich przedstawia teorię z zakresu danej metody, a następnie jej realizację praktyczną. Rozdział czwarty przedstawia najczęściej wykorzystywaną metodę syntezy – algorytm subtraktywny, który jest powszechnie wykorzystywany w syntezatorach analogowych. W rozdziale piątym zaprezentowano metodę addytywną oraz sposób generowania za jej pomocą dźwięku organów. W rozdziale szóstym omówiono syntezę dźwięku z użyciem modulacji częstotliwości (metoda FM). W rozdziale siódmym przedstawiono najbardziej skomplikowaną metodę syntezy omówioną w tej pracy – metodę modelowania matematycznego. Metoda ta wykorzystywana jest obecnie głównie w laboratoriach naukowych, gdyż daje największe możliwości uzyskania pożądanych brzmień syntezowanych dźwięków.

Rozdział ósmy jest podsumowaniem całej pracy. Omówiono w nim sposób realizacji przyjętych celów oraz przedstawiono wnioski końcowe.