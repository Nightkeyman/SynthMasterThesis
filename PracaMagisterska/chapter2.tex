\chapter{Wprowadzenie teoretyczne}\label{chapter2}

W niniejszym rozdziale przedstawiono podstawowe pojęcia związane z dziedzinami muzyki, syntezy dźwięku oraz przetwarzania sygnałów cyfrowych. Wszystkie omówione zagadnienia odnoszą się do całości pracy. Nie są one powiązane szczególnie z konkretnymi metodami syntezy.
\section{Dźwięk}

Dźwięk można rozpatrywać na dwa sposoby: fizyczny oraz muzyczny. Z fizycznego punktu widzenia, dźwięk jest zaburzeniem falowym w ośrodku sprężystym gazowym, ciekłym lub stałym, który wywołuje wrażenie słuchowe u człowieka. Poddziedzina fizyki, zajmująca się ściśle tematem dźwięku, nazywana jest akustyką. % (źródło: https://encyklopedia.pwn.pl/haslo/dzwiek;3896050.html) 
W muzyce, dźwięk rozpatrywany jest jako zjawisko, które wydobywane jest z instrumentów muzycznych lub głosu ludzkiego. Główne właściwości dźwięku to:

\begin{enumerate}
	\item wysokość dźwięku, która jest zależna od wartości częstotliwości podstawowej dźwięku,
	
	\item czas trwania - zależny od czasu produkowanego dźwięku na danym instrumencie,
	
	\item głośność, zależna od amplitudy drgań powietrza przenoszącego dźwięk, mierzona w decybelach,
	
	\item barwa dźwięku, zależna od ilości, częstotliwości składowych harmonicznych dźwięku oraz zmian ich występowania w czasie. Jako składowa harmoniczna, rozumiana jest składowa sinusoidalna dźwięku, która jest całkowitą wielkrotnością częstotliwości podstawowej danego dźwięku.
	% (źródło: https://pl.wikipedia.org/wiki/D%C5%BAwi%C4%99k_(muzyka))
\end{enumerate}

Jako pewne ustandardyzowanie dźwięków w utworach muzycznych, wprowadzono skalę dźwięków. Tradycyjną skalę tworzy osiem dźwięków. Odległości między tymi dźwiękami nazywane są interwałami, które wyraża się w jednostkach półtonów. 

Odstęp między pierwszym a ostatnim dźwiękiem w skali nazywany jest oktawą. Mierzy on 12 półtonów. Odległość ta jest wyjątkowa, gdyż dźwięk położony o oktawę dalej od pierwszego, jest jego dwukrotnością pod względem częstotliwości podstawowej składowej harmonicznej.

Istnieje również zależność między kolejnymi półtonami. Jest ona wyrażona wzorem:



\section{Rodzaje klawiatur muzycznych}
Podstawowy podział klawiatur muzycznych rozpatruje się pod względem możliwości wydawania z siebie jednego lub kilku dźwięków przy naciśnięciu kilku klawiszy na raz. Typy te nazwano klawiaturami monofonicznymi oraz polifonicznymi.


Monofonia w muzyce to faktura muzyczna utworzona z pojedynczej linii melodycznej. W danej chwili czasu utworu, powinien występować tylko jeden dźwięk, aby był on nazwany monofonicznym. W szczególnym odniesieniu do klawiatury cyfrowej lub analogowej oznacza to, iż wydobywać się z niej może się maksymalnie jeden dźwięk w jednym momencie czasu, przy naciśnięciu kilku klawiszy. Klasycznym przykładem klawiatury monofonicznej jest syntezator analogowy Minimoog.


Polifonia w muzyce oznacza natomiast występowanie kilku linii melodycznych w tym samym czasie. Polifonia klawiatury zatem oznacza możliwość wydobycia wielu dźwięków, przy naciśnięciu kilku klawiszy w tym samym czasie. Pojęcie polifonicznej klawiatury muzycznej nie precyzuje jaką ilośc wydobywających się dźwięków klawiatura powinna mieć możliwość realizacji. Cyfrowe instrumenty muzyczne na rynku posiadają ograniczoną maksymalną liczbę wydobywanych na raz dźwięków, przez skończoną moc obliczeniową procesorów.

%
% Voice allocation alghorithm - opisac jaki bedziemy uzywac
%

Jednym z głównych zadań niniejszego projektu magisterskiego jest implementacja polifonicznej klawiatury muzycznej.

\section{Przegląd metod syntezy dźwięku}
W literaturze podział metod syntezy dźwięku jest bardzo różny. Najczęściej wymieniane w literaturze to:
\begin{enumerate}
	\item metoda subtraktywna,
	
	\item metoda addytywna,
	
	\item metoda modulacji częstotliwości,
	
	\item metoda samplingowa,
	
	\item metoda modelowania matematycznego.
	
	\item metoda granulkowa
\end{enumerate}
Dwa pierwsze wymienione rodzaje syntezy zaliczane są do metod widmowych. 
Każda z powyższych metod syntezy, z wyjątkiem samplingu, posiada poświęcony jej rozdział w tej pracy. Dokładne omówienie każdej z nich będzie poruszone w tychże rozdziałach. W tym podrozdziale przedstawiono po krótce pozostałe metody syntezy, które nie będą opisywane w dalszej części pracy.

\subsection{Inne metody syntezy dźwięku}
Metoda samplingowa sprowadza się do nagrania pewnego dźwięku naturalnego, a następnie odtwarzania go w postaci cyfrowej.

Metoda granulkowa 

%http://marcdata.hamu.cz/vyzkum/dokumenty/Lit92.pdf - additive, subtractive, FM. sampling
%https://soundlab.cs.princeton.edu/publications/survey_icmc09.pdf - 4 strona, tabelka
%https://www.youtube.com/watch?v=I64y40EIPaM

\section{Klasyczne moduły syntezatorów dźwięku}
%https://www.dummies.com/art-center/music/piano/common-keyboard-terms-and-abbreviations/
%https://en.wikipedia.org/wiki/Modular_synthesizer#:~:text=Modular%20synthesizers%20are%20synthesizers%20composed,user%20to%20create%20a%20patch.
\subsection{Obwiednia dźwięku}

\section{Analogowa i cyfrowa synteza dźwięku}

\section{Protokół MIDI}
%https://www.midi.org/specifications-old/category/midi-1-0-detailed-specifications

\section{Dyskretna transformata Fouriera}
Jednym z najważniejszych narzędzi w dziedzinie cyfrowego przetwarzania sygnałów jest dyskretna transformata Fouriera. 
\subsection{Algorytm FFT}

