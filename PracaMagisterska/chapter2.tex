\chapter{Wprowadzenie teoretyczne}\label{chapter2}

W niniejszym rozdziale przedstawiono podstawowe pojęcia związane z dziedziną muzyki, syntezy dźwięku oraz przetwarzania sygnałów. 

\section{Dźwięk}

Dźwięk można rozpatrywać na dwa sposoby: fizyczny oraz muzyczny. Z fizycznego punktu widzenia, dźwięk jest zaburzeniem falowym w ośrodku sprężystym gazowym, ciekłym lub stałym, który wywołuje wrażenie słuchowe u człowieka. Poddziedzina fizyki, zajmująca się ściśle tematem dźwięku, nazywana jest akustyką. % (źródło: https://encyklopedia.pwn.pl/haslo/dzwiek;3896050.html) 
W muzyce, dźwięk rozpatrywany jest jako zjawisko, które wydobywane jest z instrumentów muzycznych lub głosu ludzkiego. Główne właściwości dźwięku to:

\begin{enumerate}
	\item wysokość dźwięku, która jest zależna od wartości częstotliwości podstawowej dźwięku,
	
	\item czas trwania - zależny od czasu produkowanego dźwięku na danym instrumencie,
	
	\item głośność, zależna od amplitudy drgań powietrza przenoszącego dźwięk, mierzona w decybelach,
	
	\item barwa dźwięku, zależna od ilości oraz częstotliwości składowych harmonicznych dźwięku. Jako składowa harmoniczna, rozumiana jest całkowita wielkrotność częstotliwości podstawowej dźwięku.
	% (źródło: https://pl.wikipedia.org/wiki/D%C5%BAwi%C4%99k_(muzyka))
\end{enumerate}

% o skalach, jak przyjeta sa wysokosci dzwieków w obecnych standardach

Istnieje zależność między kolejnymi dźwiękami przyjętymi w skali europejskiej. (opisać matematycznie)


\section{Monofonia i polifonia}
Pojęcia przedstawione w tytule niniejszego podrozdziału są powiązane z projektowaniem cyfrowych instrumentów muzycznych.
Monofonia w muzyce to faktura muzyczna utworzona z pojedynczej linii melodycznej. W danej chwili czasu utworu, powinien występować tylko jeden dźwięk, aby był on nazwany monofonicznym. W odniesieniu do klawiatury cyfrowej lub analogowej oznacza to, iż w wydobywać z niej może się maksymalnie jeden dźwięk.
Polifonia natomiast pozwala na 

\section{Przegląd metod syntezy dźwięku}

\section{Klasyczne moduły syntezatorów dźwięku}

\section{Analogowa i cyfrowa synteza dźwięku}

\section{Protokół MIDI}

\section{Dyskretna transformacja Fouriera}

