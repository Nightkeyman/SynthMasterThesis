\addcontentsline{toc}{chapter}{Wykaz ważniejszych oznaczeń i skrótów}
\chapter*{Wykaz ważniejszych oznaczeń i skrótów}

\begin{tabular}{lcl}
	\textit{DSP} & -- & ang. digital signal processor -- procesor sygnałowy, \\
	\textit{FM} & -- & ang. frequency modulation -- modulacja częstotliwości, \\
	\textit{AM} & -- & ang. amplitude modulation -- modulacja amplitudy, \\
	\textit{CS} & -- & cyfrowy syntezator, \\
	\textit{AS} & -- & analogowy syntezator, \\
	\textit{MIDI} & -- & ang. Musical Instrument Digital Interface -- cyfrowy interfejs instrumentów muzycznych, \\
	\textit{DFT} & -- & ang. Discrete Fourier Transform -- Dyskretna transformata Fouriera, \\
	\textit{VCO} & -- & ang. voltage-controlled oscillator -- oscylator kontrolowany napięciem, \\
	\textit{VCA} & -- & ang. voltage-controlled amplifier -- wzmacniacz kontrolowany napięciem, \\
	\textit{VCF} & -- & ang. voltage-controlled filter -- filtr kontrolowany napięciem, \\
	\textit{LFO} & -- & ang. low-frequency oscillation -- generator wolnych przebiegów, \\
	\textit{ADSR} & -- & ang. attack, delay, sustain, release -- generator obwiedni, \\
	\textit{DAC} & -- & ang. digital-to-analog converter -- przetwornik cyfrowo-analogowy. \\
	\textit{CPU} & -- & ang. central processing unit -- procesor. \\
	\textit{FPGA} & -- & ang. field-programmable gate array -- bezpośrednio programowalna macierz bramek. \\
	\textit{JTAG} & -- & ang. Joint Test Action Group -- protokół testowania połączeń. \\
	\textit{ISR} & -- & ang. interrupt service handler -- obsługa przerwania. \\
\end{tabular} 
